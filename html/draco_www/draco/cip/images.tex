\batchmode

\documentclass[12pt]{article}
\makeatletter

\headheight 0pt  \headsep 0pt
\topmargin 0pt  
\oddsidemargin 0pt  
\textheight 9 in \textwidth 6.5in

\title{Container-Free Numerical Algorithms in {\tt C++}}

\author{Geoffrey Furnish\\
Los Alamos National Laboratory\\
Applied Theoretical and Computational Physics Division\\
Los Alamos, NM   87545\\
{\tt furnish@lanl.gov}
}
\par\usepackage[dvips]{color}
\pagecolor[gray]{.7}


\makeatletter
\count@=\the\catcode`\_ \catcode`\_=8 
\newenvironment{tex2html_wrap}{}{} \catcode`\_=\count@
\makeatother
\let\mathon=$
\let\mathoff=$
\ifx\AtBeginDocument\undefined \newcommand{\AtBeginDocument}[1]{}\fi
\newbox\sizebox
\setlength{\hoffset}{0pt}\setlength{\voffset}{0pt}
\addtolength{\textheight}{\footskip}\setlength{\footskip}{0pt}
\addtolength{\textheight}{\topmargin}\setlength{\topmargin}{0pt}
\addtolength{\textheight}{\headheight}\setlength{\headheight}{0pt}
\addtolength{\textheight}{\headsep}\setlength{\headsep}{0pt}
\setlength{\textwidth}{468pt}
\newwrite\lthtmlwrite
\makeatletter
\let\realnormalsize=\normalsize
\global\topskip=2sp
\def\preveqno{}\let\real@float=\@float \let\realend@float=\end@float
\def\@float{\let\@savefreelist\@freelist\real@float}
\def\end@float{\realend@float\global\let\@freelist\@savefreelist}
\let\real@dbflt=\@dbflt \let\end@dblfloat=\end@float
\let\@largefloatcheck=\relax
\def\@dbflt{\let\@savefreelist\@freelist\real@dbflt}
\def\adjustnormalsize{\def\normalsize{\mathsurround=0pt \realnormalsize
 \parindent=0pt\abovedisplayskip=0pt\belowdisplayskip=0pt}\normalsize}%
\def\lthtmltypeout#1{{\let\protect\string\immediate\write\lthtmlwrite{#1}}}%
\newcommand\lthtmlhboxmathA{\adjustnormalsize\setbox\sizebox=\hbox\bgroup}%
\newcommand\lthtmlvboxmathA{\adjustnormalsize\setbox\sizebox=\vbox\bgroup%
 \let\ifinner=\iffalse }%
\newcommand\lthtmlboxmathZ{\@next\next\@currlist{}{\def\next{\voidb@x}}%
 \expandafter\box\next\egroup}%
\newcommand\lthtmlmathtype[1]{\def\lthtmlmathenv{#1}}%
\newcommand\lthtmllogmath{\lthtmltypeout{l2hSize %
:\lthtmlmathenv:\the\ht\sizebox::\the\dp\sizebox::\the\wd\sizebox.\preveqno}}%
\newcommand\lthtmlfigureA[1]{\let\@savefreelist\@freelist
       \lthtmlmathtype{#1}\lthtmlvboxmathA}%
\newcommand\lthtmlfigureZ{\lthtmlboxmathZ\lthtmllogmath\copy\sizebox
       \global\let\@freelist\@savefreelist}%
\newcommand\lthtmldisplayA[1]{\lthtmlmathtype{#1}\lthtmlvboxmathA}%
\newcommand\lthtmldisplayB[1]{\edef\preveqno{(\theequation)}%
  \lthtmldisplayA{#1}\let\@eqnnum\relax}%
\newcommand\lthtmldisplayZ{\lthtmlboxmathZ\lthtmllogmath\lthtmlsetmath}%
\newcommand\lthtmlinlinemathA[1]{\lthtmlmathtype{#1}\lthtmlhboxmathA  \vrule height1.5ex width0pt }%
\newcommand\lthtmlinlineA[1]{\lthtmlmathtype{#1}\lthtmlhboxmathA}%
\newcommand\lthtmlinlineZ{\egroup\expandafter\ifdim\dp\sizebox>0pt %
  \expandafter\centerinlinemath\fi\lthtmllogmath\lthtmlsetinline}
\newcommand\lthtmlinlinemathZ{\egroup\expandafter\ifdim\dp\sizebox>0pt %
  \expandafter\centerinlinemath\fi\lthtmllogmath\lthtmlsetmath}
\def\lthtmlsetinline{\hbox{\vrule width.1em\vtop{\vbox{%
  \kern.1em\copy\sizebox}\ifdim\dp\sizebox>0pt\kern.1em\else\kern.3pt\fi
  \ifdim\hsize>\wd\sizebox \hrule depth1pt\fi}}}
\def\lthtmlsetmath{\hbox{\vrule width.1em\vtop{\vbox{%
  \kern.1em\kern0.8 pt\hbox{\hglue.17em\copy\sizebox\hglue0.8 pt}}\kern.3pt%
  \ifdim\dp\sizebox>0pt\kern.1em\fi \kern0.8 pt%
  \ifdim\hsize>\wd\sizebox \hrule depth1pt\fi}}}
\def\centerinlinemath{%\dimen1=\ht\sizebox
  \dimen1=\ifdim\ht\sizebox<\dp\sizebox \dp\sizebox\else\ht\sizebox\fi
  \advance\dimen1by.5pt \vrule width0pt height\dimen1 depth\dimen1 
 \dp\sizebox=\dimen1\ht\sizebox=\dimen1\relax}

\def\lthtmlcheckvsize{\ifdim\ht\sizebox<\vsize\expandafter\vfill
  \else\expandafter\vss\fi}%
\makeatletter \tracingstats = 1 


\begin{document}
\pagestyle{empty}\thispagestyle{empty}%
\lthtmltypeout{latex2htmlLength hsize=\the\hsize}%
\lthtmltypeout{latex2htmlLength vsize=\the\vsize}%
\lthtmltypeout{latex2htmlLength hoffset=\the\hoffset}%
\lthtmltypeout{latex2htmlLength voffset=\the\voffset}%
\lthtmltypeout{latex2htmlLength topmargin=\the\topmargin}%
\lthtmltypeout{latex2htmlLength topskip=\the\topskip}%
\lthtmltypeout{latex2htmlLength headheight=\the\headheight}%
\lthtmltypeout{latex2htmlLength headsep=\the\headsep}%
\lthtmltypeout{latex2htmlLength parskip=\the\parskip}%
\lthtmltypeout{latex2htmlLength oddsidemargin=\the\oddsidemargin}%
\makeatletter
\if@twoside\lthtmltypeout{latex2htmlLength evensidemargin=\the\evensidemargin}%
\else\lthtmltypeout{latex2htmlLength evensidemargin=\the\oddsidemargin}\fi%
\makeatother
\stepcounter{section}
\stepcounter{section}
\stepcounter{section}
\stepcounter{section}
\stepcounter{section}
{\newpage\clearpage
\lthtmlfigureA{figure525}%
\begin{figure}\begin{tex2html_preform}\begin{verbatim}// cip_traits.hh

template<class T>
class cip_traits
{
// We actually don't put /anything/ in here!
};\end{verbatim}\end{tex2html_preform}
\end{figure}%
\lthtmlfigureZ
\hfill\lthtmlcheckvsize\clearpage}

{\newpage\clearpage
\lthtmlfigureA{figure595}%
\begin{figure}\begin{tex2html_preform}\begin{verbatim}// umat_traits.hh
// Shows specialization of cip_traits<T> for a particular class.

#include "cip_traits.hh"
#include "UserMat.hh"

// Partially specialize cip_traits for the template class UserMat<T>.

template<class T>
class cip_traits< UserMat<T> >
{
  public:
    typedef T NumT;
    static int begin_row_index( const UserMat<T>& m ) { return 0; }
    static int nx( const UserMat<T>& m ) { return m.number_of_rows(); }
// etc...
};\end{verbatim}\end{tex2html_preform}
\end{figure}%
\lthtmlfigureZ
\hfill\lthtmlcheckvsize\clearpage}

\stepcounter{section}
{\newpage\clearpage
\lthtmlfigureA{figure625}%
\begin{figure}\begin{tex2html_preform}\begin{verbatim}// vwrap.hh
// An adaptor class which wraps an std::vector so that it presents an
// interface useable by ltsolve3.

#include <vector>

template<class T>
class vwrap
{
    std::vector<T>& v;
  public:
    typedef std::vector<T>::reference reference;
    typedef std::vector<T>::const_reference const_reference;

    vwrap( std::vector<T>& vv ) : v(vv) {}
    reference operator()(int n) { return v[n]; }
    const_reference operator()(int n) const { return v[n]; }
};\end{verbatim}\end{tex2html_preform}
\end{figure}%
\lthtmlfigureZ
\hfill\lthtmlcheckvsize\clearpage}

{\newpage\clearpage
\lthtmlfigureA{figure668}%
\begin{figure}\begin{tex2html_preform}\begin{verbatim}// carraywrap.hh
// An adaptor class which wraps a C style array.

template<class T>
class carraywrap
{
    T **v;
    int row_offset;
    int n_rows;
    int col_offset;
    int n_cols;
  public:
    carraywrap( T **vv, int ro, int nr, int co, int nc )
        : v(vv),
          row_offset( ro ), n_rows( nr ),
          col_offset( co ), n_cols( nc )
    {}
    T& operator()(int r, int c) { return v[r][c]; }
    const T& operator()(int r, int c) const { return v[r][c]; }

// Accessors:
    int nx() const { return n_rows; }
// etc...
};\end{verbatim}\end{tex2html_preform}
\end{figure}%
\lthtmlfigureZ
\hfill\lthtmlcheckvsize\clearpage}

\stepcounter{section}
\stepcounter{section}
\stepcounter{section}

\end{document}
