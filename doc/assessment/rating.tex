\documentclass{article}
\title{Six Software Subcultural Patterns:\\ Six Ways To Skin A Cat}
\author{Mark G. Gray}
\begin{document}
\maketitle

Various authors have classified software subcultural patterns in
various ways.  Since each view provides a different way of looking at
the subculture, each is useful in the task of classifying where we are.

Here are the various levels of software
subcultures\footnote{G. M. Weinberg \em{Quality Software Management
Volume 1} (New York: Dorset House Publishing, 1991) pg 22-23}

\begin{center}
\begin{tabular}{r|l|l|l|l|r}
\parbox[t]{3em}{CMM\\ Level} & \parbox[t]{14ex}{Crosby's\\ Management
Attitudes} & \parbox[t]{14ex}{Humphrey's\\ Process Types} &
\parbox[t]{18ex}{Curtis's\\ Personnel\\ Treatment} &
\parbox[t]{14ex}{Weinberg's\\ Congruence\\ Degree} & SPI \\ \hline
  &             &         &        & Oblivious&   0 \\
1 & Uncertainty   & Initial    & Herded & Variable & 1-2 \\
2 & Awakening     & Repeatable & Managed & Routine & 3-4 \\
3 & Enlightenment & Defined    & Tailored & Steering & 5-6 \\
4 & Wisdom        & Managed    & Institutionalized & Anticipating & 7-8
\\
5 & Certainty     & Optimizing & Optimized & Congruent & 9-10 
\end{tabular}
\end{center}

Here is how Weinberg summarizes his classification:
\begin{description}
\item[Oblivious: ] We don't even know that we're performing a process.
\item[Variable: ] We do whatever we feel like at the moment.
\item[Routine: ] We follow our routines (except when we panic).
\item[Steering: ] We choose among our routines by the results they produce.
\item[Anticipating: ] We establish routines based on our past
experience with them. 
\item[Congruent: ] Everyone is involved with improving everything all
the time.
\end{description} 
\end{document}