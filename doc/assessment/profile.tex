%  ========================================================================  %
% 
%       Author: Mark G. Gray
%               Los Alamos National Laboratory
%       Date:   Wed Feb  3 12:20:05 MST 1999
% 
%       Copyright (c) 1999 U. S. Department of Energy. All rights reserved.
% 
%       $Id$
% 
%  ========================================================================  %
\documentstyle{article}
\title{Software Process Assessment\\ Project Profile Sheet}
\pagestyle{myheadings}
\markright{Draco 99 Project Profile}
\begin{document}
\maketitle
\begin{enumerate}
\item Project Name/Acronym: Draco 99
\item Project Type/Application Domain: Software Development for
Radiation Transport Support 
\item Customer: DOE
\item Location of Primary Project Effort: Los Alamos National
Laboratory
\item Current Phase: Implementation, 2nd pass of spiral
\item Start Date: Oct 1998
\item End Date: Oct 1999
\item Project Team and Individual Responsibilities:
\begin{center}
\begin{tabular}{l|l}
Personnel & Responsibility \\ \hline
John McGhee & Project Lead \\
Denise Archuleta & Project Planner \\
Mark Gray & Software Quality Assurance \\
Randy Roberts & Deterministic Transport \\
Shawn Pautz & Deterministic Transport \\
Tom Evans & Monte Carlo Transport \\
Todd Urbatsch & Monte Carlo Transport \\
\end{tabular}
\end{center}
\item Programming Languages: Currently--
\begin{center}
\begin{tabular}{l|r}
Language & \% use \\ \hline
{\tt C++} & 88 \\
{\tt Python} & 9 \\
{\tt Perl} & 2 \\
{\tt Expect} & 1
\end{tabular}
\end{center}
\item Size of Project: Currently--
\begin{center}
\begin{tabular}{l|r}
Language & Lines of Code \\ \hline
{\tt C++} & 27,406 \\
{\tt Python} & 2,959 \\
{\tt Perl} & 506 \\
{\tt Expect} & 318
\end{tabular}
\end{center}
\item Cost: 3 FTEs
\item Target Platform Description: Sun workstations running Solaris,
SGI workstations and SMP's running Irix
\item Development Platform Description: Sun workstations running Solaris,
SGI workstations and SMP's running Irix
\item Applicable Standards:
\begin{center}
\begin{tabular}{l|p{3in}}
Stage & Document \\ \hline
Requirements & Daniel P. Freedman and Gerald M. Weinberg, {\em
              Exploring Requirements: Quality Before Design}, New
              York, NY: Dorset House Publishing, 1989\\ 
Design      & John Lakos, {\em Large Scale C++ Software Design},
              Reading, MA: Addison-Wesley Publishing, 1996;
              Matthew H. Austern, {\em Generic Programming and the
              STL}, Reading, MA: Addison-Wesley Publishing, 1998;
              Bertrand Meyer, {\em Object-Oriented Software
              Construction}, Upper Saddle River, NJ: Prentice Hall
	      PTR, 1997\\
Code        & ISO/IEC 14882; GNU Coding Standards (9 September 1996); 
	      POSIX 199309; Scott Meyers, {\em Effective C++},
              Reading, MA: Addison-Wesley Publishing, 1997; Scott
              Meyers, {\em More Effective C++}, Reading, MA:
              Addison-Wesley Publishing, 1996 \\
Test        & Analytic Test Method; LA-UR-98-5735; LA-1709; LAMS-2421;
              LA-UR-96-1799 \\ 
All         & Daniel P. Freedman and Gerald M. Weinberg, {\em Handbook
              of Walkthroughs, Inspections, and Technical Reviews:
              Evaluating programs, Projects, and Products}, New
              York, NY: Dorset House Publishing, 1990;
              {\em Planning Innovative Projects: A course to launch
              projects and teams, customized for Los Alamos National
              Laboratory}, Erika Jones \& Associates, Inc.; 
              {\em Mastering Projects Workshop}, True North pgs, Inc.
\end{tabular}
\end{center}
\item Brief Functional Description: The Draco Computational Physics
System (DCPS) is a radiation-transport oriented software-engineering
system within the Transport Methods Group (X-TM) in the Applied
Theoretical and Computational Physics Division (X Division) at Los
Alamos National Laboratory (LANL). The Draco system is designed to
create a stable, convenient, well planned environment for radiation
transport code development. Draco is a comprehensive radiation
transport framework that provides key reusable components for both
serial and parallel computational physics codes.
\end{enumerate}
\end{document}

%  ========================================================================  %
