%%---------------------------------------------------------------------------%%
%% vendors.tex
%% Time-stamp: <99/01/26 11:25:45 tme>
%% description of vendor libraries
%%---------------------------------------------------------------------------%%

\chapter{Vendor Libraries}
\label{app:vendor_libs}

As described in \S~\ref{sec:draco_dependencies}, \draco\ uses and
requires several external vendor libraries.  The packages that require
these vendors are listed in Table~\ref{tab:vendor}.  This appendix
gives additional details on the vendor libraries.  Specifically
included are common headers used by \draco\ and link-line
dependencies.

%%---------------------------------------------------------------------------%%

\section{MPI}
\label{appsec:mpi}

\soft{MPI} is the message passing interface used for most parallel codes including \draco.
While \draco\ can be built without \soft{MPI}, we find that most production capabilities require \soft{MPI} and the scalar build mode becomes useful only for debugging.  At the current time, \draco\ can use either the \href{http://www.openmpi.org}{OpenMPI} or \href{http://www.mcs.anl.gov/research/projects/mpich2/}{MPICH2} flavor of \soft{MPI}. The \draco\ build system should automatically detect the available \soft{MPI} system and link against it (enabling \comp{DRACO\_C4=MPI}).  \index{OpenMPI}\index{MPICH2}\index{build option!DRACO\_C4}

The \draco\ development environment does not normally provide the \soft{MPI} vendor installation since it must be compiled to take advantage of specialized hardware on each machine (interconnects, etc).  The one exception is on CCS development machines were a version is provided for testing purposes only. 

\soft{MPI} is discovered by the build system via a call to \comp{vendor\_libraries.cmake} which, in turn, calls the \cmake\ provided \comp{FindMPI.cmake} (use \comp{cmake --help-module FindMPI} for detailed information).  If the build system cannot find the flavor of \soft{MPI} that you are targeting, you can try manually set the \cmake\ variables \comp{MPIEXEC} and \comp{MPIEXEC\_NUMPROC\_FLAG}.  This must be done on platforms like Cielo.

\subsection{Additional Notes}
\begin{itemize}
\item On LANL HPC machines, you must link against a \soft{MPI}  built with compatible compilers.  That is you may need to load a gcc version or an Intel version of openmpi using the module command.
\item Some vendors (e.g.: \soft{Trilinos}, \soft{ScaLAPACK}) also use \soft{MPI} and you should be careful to configure your code to use the same version of \soft{MPI} as the vendor libraries. 
\item To link your component against MPI, modify the \comp{target\_link\_libraries} command to include \comp{MPI\_LIBRARIES}.
\item The MPI header directory is provided by \comp{MPI\_INCLUDE\_PATH}.
\end{itemize}

%%---------------------------------------------------------------------------%%

\section{LAPACK}
\label{appsec:lapack} \index{LAPACK} \index{BLAS}

The \href{http://www.netlib.org/lapack/}{LAPACK} libraries provide efficient implementation of linear algebra operations on shared-memory vector and parallel processors. The \draco\ development environment provides it's own installation of \soft{LAPACK} that can be loaded into the developer environment through the use of the \draco\ modules feature.

The \draco\ build system uses the \cmake\ provided \comp{FindLAPACK.cmake} module (use \comp{cmake --help-module FindLAPACK} for more details) to discover LAPACK.  If the build system is finding the wrong installation of LAPACK, you can manually set \comp{BLAS\_blas\_LIBRARY} and \comp{LAPACK\_lapack\_LIBRARY} to the full path of these 2 libraries before running \cmake\ for the first time.

Make your component depend on LAPACK by adding \comp{LAPACK\_LIBRARIES} to your \comp{target\_link\_libraries} command.

%%---------------------------------------------------------------------------%%

\section{GSL, GNU Scientific Library}
\label{appsec:gsl}\index{GNU Scientific Library}

The \href{http://www.gnu.org/software/gsl/}{GSL} libraries provide efficient implementation of common numerical operations like root-finding, splines, and statistics. The \draco\ development environment provides it's own installation of \soft{GSL} that can be loaded into the developer environment through the use of the \draco\ modules feature.

The \draco\ build system uses its homegrown \comp{FindGSL.cmake} module to discover GSL.  If the build system is finding the wrong installation of GSL, you can manually set \comp{GSL\_INC\_DIR} and \comp{GSL\_LIB\_DIR} to the full paths of these two items before running \cmake\ for the first time.

Make your component depend on GSL by adding \comp{GSL\_LIBRARIES} to your \comp{target\_link\_libraries} command.  You may also need to add \comp{GSL\_INCLUDE\_DIRS} to your list of \comp{include\_directories}.

%%---------------------------------------------------------------------------%%

\section{CUDA}
\label{appsec:cuda} \index{CUDA}

\href{http://www.nvidia.com/object/cuda_home_new.html}{CUDA}$^{TM}$ is a parallel computing platform and programming model invented by NVIDIA. It enables dramatic increases in computing performance by harnessing the power of the graphics processing unit (GPU).  Support for CUDA is still experimental and you will need to review the code found in the \pkg{device} component, the discovery routines in \comp{vendor\_libraries.cmake} and the \cmake\ module \comp{FindCUDA} for details on using this vendor.

%%---------------------------------------------------------------------------%%

\section{DaCS}
\label{appsec:dacs} \index{DaCS}

DaCS is the communication library used to run code on the IBM Cell architecture (Roadrunner).  Support for DaCS is mature, but no further development is planned since this architecture has been discontinued.  You can glean some of the implementation features by review \draco's \pkg{device} component.

%%---------------------------------------------------------------------------%%

\section{XMGRACE}
\label{appsec:grace} \index{Grace}

Grace is a WYSIWYG 2D plotting tool for the X Window System.  The \draco\ development environment provides it's own installation of \soft{Grace} that can be loaded into the developer environment through the use of the \draco\ modules feature.

The \draco\ build system uses its homegrown \comp{FindGrace.cmake} module to discover Grace.  If the build system is finding the wrong installation of Grace, you can manually set \comp{GRACE\_INC\_DIR} and \comp{GRACE\_LIB\_DIR} to the full paths of these two items before running \cmake\ for the first time.

Make your component depend on Grace by adding \comp{Grace\_LIBRARIES} to your \comp{target\_link\_libraries} command.  You may also need to add \comp{Grace\_INCLUDE\_DIRS} to your list of \comp{include\_directories}.

%%---------------------------------------------------------------------------%%
\section{EOSPAC}
\label{appsec:eospac} \index{EOSPAC}

EOSPAC is a LANL provided library that provides access to Sesame data.  \draco\ provides a module that can be loaded to set the default developer environment for building with EOSPAC.  \pkg{cdi\_eospac} requires this vendor.

The \draco\ build system uses its homegrown \comp{FindEOSPAC.cmake} module to discover EOSPAC.  If the build system is finding the wrong installation of EOSPAC, you can manually set \comp{EOSPAC\_INC\_DIR} and \comp{EOSPAC\_LIB\_DIR} to the full paths of these two items before running \cmake\ for the first time.

Make your component depend on EOSPAC by adding \comp{EOSPAC\_LIBRARY} to your \comp{target\_link\_libraries} command.  You may also need to add \comp{EOSPAC\_INCLUDE\_DIRS} to your list of \comp{include\_directories}.

%%---------------------------------------------------------------------------%%
\section{Random123}
\label{appsec:random123} \index{Random123}

Random123 is a header-only random number generation implementation used by the Jayenne project. \draco\ provides a module that can be loaded to set the default developer environment for building with Random123.  \pkg{rng} requires this vendor.

The \draco\ build system uses its homegrown \comp{FindRANDOM123.cmake} module to discover Random123.  If the build system is finding the wrong installation of Random123, you can manually set \comp{RANDOM123\_INC\_DIR} to the full path before running \cmake\ for the first time.

Make your component depend on Random123 by adding \comp{RANDOM123\_INCLUDE\_DIRS} to your list of \comp{include\_directories}.

%%%---------------------------------------------------------------------------%%
%
%\section{SHMEM}
%\label{appsec:shmem}
%
%%%---------------------------------------------------------------------------%%
%
%\section{SPRNG}
%\label{appsec:sprng}
%
%%%---------------------------------------------------------------------------%%
%
%\section{PCGLIB}
%\label{appsec:pcglib}