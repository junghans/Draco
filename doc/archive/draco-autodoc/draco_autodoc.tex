%%---------------------------------------------------------------------------%%
%% draco_autodoc.tex
%% Mike Buksas
%% $Id$
%%---------------------------------------------------------------------------%%
\documentclass[memo]{ResearchNote}
\usepackage[centertags]{amsmath}
\usepackage{amssymb,amsthm,graphicx}
\usepackage[mathcal]{euscript}
\usepackage{tmadd,tmath}
\usepackage{cite}
\usepackage{latexmake}

%%---------------------------------------------------------------------------%%
%% DEFINE SPECIFIC ENVIRONMENTS HERE
%%---------------------------------------------------------------------------%%
%\newcommand{\elfit}{\ensuremath{\operatorname{Im}(-1/\epsilon(\vq,\omega)}}
%\msection{}-->section commands
%\tradem{}  -->add TM subscript to entry
%\ucatm{}   -->add trademark footnote about entry

%%---------------------------------------------------------------------------%%
%% BEGIN DOCUMENT
%%---------------------------------------------------------------------------%%
\begin{document}

%%---------------------------------------------------------------------------%%
%% OPTIONS FOR NOTE
%%---------------------------------------------------------------------------%%

\toms{Distribution}
%\toms{Joe Sixpak/XTM, MS B226}
\refno{CCS-4:04-35(U)}
\subject{Using the Draco Autodoc System}

%-------NO CHANGES
\divisionname{Computer and Computational Sciences Division}
\groupname{CCS-4:Transport Methods Group}
\fromms{Mike Buksas/CCS-4 D409}
\phone{(505)667--7580}
\originator{mwb}
\typist{mwb}
\date{\today}
%-------NO CHANGES

%-------OPTIONS
%\reference{NPB Star Reimbursable Project}
%\thru{P. D. Soran, XTM, MS B226}
%\enc{list}      
%\attachments{list}
%\cy{list}
%\encas
%\attachmentas
%\attachmentsas 
%-------OPTIONS

%%---------------------------------------------------------------------------%%
%% DISTRIBUTION LIST
%%---------------------------------------------------------------------------%%

\distribution {
  Kent G. Budge, CCS-4, D-409 \\
  Michael Buksas,  CCS-4, D-409 \\
  Jeff Densmore,  CCS-4, D-409 \\
  Thomas Evans,  CCS-4, D-409 \\
  Robert B. Lowrie, CCS-2, D-413 \\
  Jim E. Morel, CCS-2, D-413 \\
  John A. Turner, CCS-2, D-413 \\
  Kelly Thompson,  CCS-4, D-409 \\
  Todd Urbatsch,  CCS-4, D-409 \\
  Jim Warsa,  CCS-4, D-409 \\
}

%%---------------------------------------------------------------------------%%
%% BEGIN NOTE
%%---------------------------------------------------------------------------%%

\opening

\section{Overview}

The Draco Autodoc System uses doxygen to automatically create
documentation for code projects which conform to the Draco source-code
layout. The system generates both package-level and component-level
documentation. Currently, only html output of the documentation is
fully supported, with a \LaTeX\ version forthcoming.

\section{Source Files for Autodoc}

In this section we detail the necessary files which autodoc will use
to configure and control the appearence of the product. We break this
down into the package and component level documentation.

A Draco-style software project is a single package consisting of
several components. The relevant directories in the source-code layout
are as follows:

\begin{tabbing}
XX\=XX\=XX\= \kill
{\em package}/ \\
\> autodoc/ \\
\> \> html/ \\
\> config/  \\
\> src/     \\
\> {\em component-A}/ \\
\> \> autodoc/    \\
\> {\em component-B}/ \\
\> \> autodoc/    \\
\end{tabbing}

\subsection{Packages}

The configuration process will look for the following files in {\tt
  {\em package}/autodoc}.

\begin{itemize}
\item configure.ac
\item mainpage.dcc.in
\item html/header.html.in
\item html/footer.html.in
\end{itemize}

The {\tt configure.ac} file can be be templated on the one in {\tt
  draco/autodoc}. Most of the work is done with a single macro: {\tt
  AC\_PACKAGE\_AUTODOC} .

The file {\tt mainpage.dcc.in} should include the Doxygen command
\verb=\mainpage=. This will cause 

The header and footer are specific to the html documentation and hence
live in their own directory. The resulting {\tt header.html} and {\tt
  footer.html} will be used in the construction of all html files for
the package and all of it's components. (There is currently no way to
replace these files for a component)

Other files required by either the header (e.g. image files, style
sheets) or the mainpage (images, etc..) should be placed in the {\tt
  autodoc} or {\tt autodoc/html} directories as appropiate. These
supplemental files will all be copied to the directory in which they
need to appear.

Since the header and footer files are straight html, referring to
images is done in the usual fashion. To reference an image in other
doxygen comment blocls, including the mainpage, use the \verb=\image=
command:

\begin{verbatim}
  \image html stdcell.jpg "Cell definitions"
\end{verbatim}
This will insert the image into the html output.  See the doxygen
manual for more information.

The configuration procedure makes some variables available which are
useful in the mainpage or headers. See the files in {\tt
  draco/autodoc} for examples.

\subsection{Components of Packages}

The process for configuring each component of a package for
autodocumentation is similar to the package level documentation. In
this case, the configuration is performed from the source directory of
the component so a seperate {\tt configure.ac} file is not needed. The
macro {\tt AC\_DRACO\_AUTODOC()} performs all of the necessary
configuration. The single argument to this macro is the name and
version number (if desired) of the parent package.

A component is autdoc'd only if it has an {\tt autodoc} subdirectory
in the source tree. This directory should contain (at least) a .dcc
file containing the doxygen \verb=\mainpage= command in a
comment. This will produce the main page for the component. This
directory will also contain any supplemental images required by the
mainpage or any of the other generated documentation. As with the
package-level files, these will be copied into the destination directory.

\section{Output}

By default, the autodoc output for the package goes to {\tt
  \$\{prefix\}/documentation/html} where {\tt \$\{prefix\}} is the
project build directory. (LaTeX output, when fully supported, will go
to {\tt /latex}). Component autodoc output goes to {\tt
  \$\{prefix\}/documentation/html/{\em component}}.

Note that if more than one project's documentation is built with the
same value of {\tt \$\{prefix\}} clobbering will occur. In this case
the base directory of the output ({\tt \$\{prefix\}/documentation})
can be changed with the configure option {\tt --with-doc-prefix=path}.
 
\clearpage
\closing
\end{document}

%%---------------------------------------------------------------------------%%
%% end of draco_autodoc.tex
%%---------------------------------------------------------------------------%%
