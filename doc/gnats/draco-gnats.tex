%%---------------------------------------------------------------------------%%
%% draco-gnats.tex
%% John McGhee
%% $Id$
%%---------------------------------------------------------------------------%%
\documentclass[11pt]{nmemo}
\usepackage[centertags]{amsmath}
\usepackage{amssymb,amsthm,graphicx}
\usepackage[mathcal]{euscript}
\usepackage{tmadd,tmath}
\usepackage{cite}

%%---------------------------------------------------------------------------%%
%% DEFINE SPECIFIC ENVIRONMENTS HERE
%%---------------------------------------------------------------------------%%
%\newcommand{\elfit}{\ensuremath{\operatorname{Im}(-1/\epsilon(\vq,\omega)}}
%\msection{}-->section commands
%\tradem{}  -->add TM subscript to entry
%\ucatm{}   -->add trademark footnote about entry

\newcommand{\draco}{{\normalfont\sffamily Draco}}
\newcommand{\xemacs}{{\normalfont\bfseries XEmacs}}
\newcommand{\gnats}{{\normalfont\bfseries GNATS}}

%%---------------------------------------------------------------------------%%
%% BEGIN DOCUMENT
%%---------------------------------------------------------------------------%%
\begin{document}

%%---------------------------------------------------------------------------%%
%% OPTIONS FOR NOTE
%%---------------------------------------------------------------------------%%

\toms{Distribution}
%\toms{Joe Sixpak/XTM, MS B226}
\refno{XTM:JMM-99-19 (U)}
\subject{\gnats\ Bug Tracking System}

%-------NO CHANGES
\divisionname{Applied Theoretical \& Computational Physics Div.}
\groupname{X-TM:Transport Methods Group}
\fromms{John McGhee/XTM, MS D409}
\phone{(505)667-9552}
\originator{jmm}
\typist{jmm}
\date{\today}
%-------NO CHANGES

%-------OPTIONS
%\reference{NPB Star Reimbursable Project}
%\thru{P. D. Soran, XTM, MS B226}
%\enc{list}      
%\attachments{list}
%\cy{list}
%\encas
%\attachmentas
%\attachmentsas 
%-------OPTIONS

%%---------------------------------------------------------------------------%%
%% DISTRIBUTION LIST
%%---------------------------------------------------------------------------%%

\distribution {
J.M. McGhee, XTM MS D409\\ 
T.M. Evans, XTM MS D409\\ 
M.G. Gray, XTM MS D409\\ 
S.D. Pautz, XTM MS D409\\ 
R.M. Roberts, XTM MS D409\\ 
T.J. Urbatsch, XTM MS D409\\ 
J.S. Warsa, XTM MS D409\\
}

%%---------------------------------------------------------------------------%%
%% BEGIN NOTE
%%---------------------------------------------------------------------------%%

\opening

The GNU Public License bug-tracking system \gnats\ has been 
installed on /n/skiathos for use by the \draco\ team. This memo
describes how to set up your local environment to access the 
system, and how to use it.


\section{Introduction}

\gnats\ is a software bug tracking system available under
the GNU Public License. Several user references are available.
A user's manual is available on-line at:
\begin{verbatim} http://sourceware.cygnus.com/gnats/ \end{verbatim}
The \gnats\ info pages can be accessed from \xemacs\ via ``C-h i g gnats'',
after performing the INFOPATH setup described below.
The \draco\ \gnats\ administrator is John McGhee (mcghee@lanl.gov).
Questions concerning \gnats\ may be directed to John.

This memo is aimed at new users.
The file \$\{GNATS\_DIR\}/README
contains additional installation and administration notes.

\section{Set-up for New \draco\ \gnats\ Users}

The following changes should be made to setup \gnats\ for
machines on the radtran LAN:

\subsection {Changes to the Shell Initialization File}

The example below assumes you are using the bash shell.
The syntax should be adjusted as required for other shells.
Add the following text or it's equivalent to your .bashrc file.

\begin{verbatim}
# Setup variables for the GNATS bug-tracking system.
  GNATS_DIR=/n/skiathos/radtran/gnats
  GNATS_PATH=${GNATS_DIR}/$(uname)/bin:${GNATS_DIR}/$(uname)/libexec/gnats
  GNATS_MANPATH=${GNATS_DIR}/man
  GNATS_INFOPATH=${GNATS_DIR}/info
  export GNATS_ELISP_DIR=${GNATS_DIR}/share/emacs/lisp/$(uname)
  export GNATS_ROOT=${GNATS_DIR}/lib/gnats/gnats-db

# Set your path(s) so that the radtran GNATS utilities are found
# before the GNATS utilities that may (are) included with your XEmacs.
  export PATH=${GNATS_PATH}:${PATH}
  export MANPATH=${GNATS_MANPATH}:${MANPATH}
  export INFOPATH=${GNATS_INFOPATH}:${INFOPATH}
\end{verbatim}
%$

\subsection {Changes to the \xemacs\ Initialization File:}

Add the following text or it's equivalent to your .emacs file.
 \begin{verbatim}
;; Set the load-path so that the radtran GNATS elisp files preempt
;; (shadow) the GNATS elisp files that come with XEmacs, GNU, et al.
  (setq my-gnats-dir (getenv "GNATS_ELISP_DIR"))
  (setq my-home-dir  (getenv "HOME"))
  (setq my-elisp-dir (concat my-home-dir "/lib/elisp"))
  (setq load-path (cons my-elisp-dir (cons my-gnats-dir load-path )))

;; Load the various functions that setup the GNATS bug tracking system.
;; Note that this is not necessary if "draco-rtt" is loaded by your .emacs
  (autoload 'edit-pr "gnats"
            "Command to edit a problem report." t)
  (autoload 'view-pr "gnats"
            "Command to view a problem report." t)
  (autoload 'gnats-mode "gnats"
            "Major mode for editing of problem reports." t)
  (autoload 'query-pr "gnats"
            "Command to query information about problem reports." t)
  (autoload 'summ-pr "gnats"
            "Command to display a summary listing of problem reports." t)
  (autoload 'change-gnats "gnats"
            "Change the GNATS database in use." t)
  (autoload 'send-pr-mode "send-pr"
            "Major mode for sending problem reports." t)
  (autoload 'send-pr "send-pr"
            "Command to create and send a problem report." t)

\end{verbatim}

\section{Reporting and Editing Bugs }

In general, \gnats\ allows the submission through e-mail of PR's (Problem Reports)
in any of several user-defined categories (listed below). Problems 
are further classified  to 
be software bugs, documentation bugs, a request for a change in behavior,
or a support problem or question. A PR consists of a form with
various "fields" to be filled in. Every PR is originally given a
state of "open". Other available states are analyzed, feedback, suspended, and
closed. Analyzed means that the responsible party has looked at the
PR and produced a preliminary evaluation of what needs to be done
to fix it. Feedback means the problem has been fixed and the responsible
party is waiting on the originator to acknowledge that the fix is OK.
Suspended means that work on the problem has been postponed.
Closed is where PR's go to die. Once entered into the system, a PR
is permanently available for perusal or editing, regardless of it's status.


\gnats\ is set up to run on any SunOS or IRIX64 on the radtran LAN.
The best way to interface with the system is to
use \xemacs. 
 There are three major commands that you will want to use:
``M-x send-pr'', ``M-x edit-pr'', and ``M-x query-pr''. A brief synopsis
is provided below. See the \gnats\ manual for additional details.

\subsection{M-x send-pr }
Send-pr brings up a window for sending in a PR. You can actually 
edit any field in the PR manually but it is safer to use the \xemacs\ mini-buffer.
The mini-buffer provides defaults, command completion, and won't accept bad
data. ``C-c C-f'' edits the field the cursor happens to be over.
Send-pr invokes the ``send-pr mode''.
Use ``C-h m'' while in the send-pr buffer to get mode help for 
other key bindings.
When you are done, if
you want to send in the PR, C-c C-c checks the
PR for errors and transmits the PR. Lines beginning with
``SEND\_PR:'' and text between ``$<$'' and ``$>$'' are automatically stripped.
If there are errors, your PR
goes into a separate buffer where you can work on it some more,
otherwise, you get a ``PR sent'' message and subsequently,
you will get an e-mail notifying you that the PR has been accepted,
giving you a unique PR number, and the name of the individual assigned
responsibility for the PR.  This process generally takes about
ten  minutes, since the \gnats\ daemon
only checks that often for incoming PR's.
Keep track of the number \gnats\ assigns to
your problem. Your PR is the only PR that has, or ever will
have, this number as far as \gnats\ is concerned. When you communicate
to \gnats\ about a PR, this number is required to ID the PR in question.

\subsection{M-x edit-pr }
Edit-pr works just about the same way as send-pr above. Do NOT edit
the PR number, and you should really avoid changing historical info
like Description, How-to-Repeat, etc.. Do a C-c C-c when done to
transmit the changes. Edit-pr invokes the ``gnats mode''.
Use ``C-h m'' while in the edit-pr buffer 
to get mode help for other key bindings.
If you change certain fields in the PR, \xemacs\ will follow
with a mail window to concerned parties. Edit in this window as you
like, and C-c C-c to send the e-mail. Although the addresses should
be set pretty much automatically, watch out for "local"  To:
and From: addresses. i.e.
"mcghee" instead of "mcghee@lanl.gov". Your mail may handle some of
these OK, or it might not. Always use a complete address to be sure.

\subsection{M-x query-pr }
Queries the \gnats\ database looking for PR's that match the criteria
that you specify. For example, to get back a list of all the open PR's
in the draco-units category, use the options:
\begin{verbatim} "--category=draco-units --state=open" \end{verbatim}
 You can get a list of the available search options using: 
\begin{verbatim} "query-pr --help" \end{verbatim}
GNU regular expression syntax is supported.
For example: \begin{verbatim} --state="o|a" \end{verbatim}
finds PR's whose state field begins
with "o" or "a" (open, or analyzed). For more details, see the manual.

\subsection{Direct E-mail }
If you send e-mail to bugs@skiathos.lanl.gov with
"my-cat/my-num" in the subject, the message will be appended to
the unique \gnats\ bug report number "my-num" which should
be found in the category "my-cat". 
For example, sending e-mail to
bugs@skiathos.lanl.gov with "Subject: draco-units/36 " will
append the e-mail to problem report 36, which should be found in the 
draco-units category.
The usual usage is to cc bugs@skiathos.lanl.gov on any e-mail that
you would like to be made a part of the PR record.

\subsection {PR Categories}
The current list of accepted categories is given below. This can be altered 
at any time by your friendly \gnats\ administrator.
Do not submit anything to the "pending" category, that is for
\gnats\ administration internal use only. There is also a ``gnats-test''
category that can be used for examples and training.
Note that each category has a "responsible" individual.  This
individual gets a copy of the PR when it is received by \gnats\, and also
gets a copy of any changes that are made by others. The last field is
a list of other people to notify of significant events in the life
of a PR.

\begin{verbatim}
# Categories
# category:description:responsible:others to e-mail 
#######################################################################
#
# Draco Clients:
#
solon:generic solon problems:roberts:pautz
tycho:generic tycho problems:mcghee:
mach:generic mach problems:lowrie:
milagro:generic milagro problems:evans:urbatsch
milstone:generic milstone problems:evans:urbatsch
kepler:generic kepler problems:evans:urbatsch
#
#######################################################################
#
# Draco:
#
draco:generic draco problems:mcghee:rsqrd,evans
draco-c4:draco C4 communications package:furnish:
draco-ds++:draco ds++ services package:furnish:
draco-xyz-mt:draco XYZ mesh type:pautz:
draco-matprops:draco material properties:roberts:
draco-timestep:draco time-step controller:mcghee:
draco-xm:draco XM expression template engine:furnish:
draco-units:draco units package:roberts:
draco-radphys:draco radiation physics utilities:roberts:
draco-traits:draco C++ container traits utility:roberts:
draco-nml:draco name-list utility:furnish:
draco-rng:draco random number generator utility:evans:
draco-imc:draco implicit Monte-Carlo utilities:evans:
draco-diffusion:draco diffusion package:roberts:
draco-elisp:Draco elisp code development environment:evans:
draco-dejagnu:draco dejagnu testing framework:mcghee:
draco-build-system:draco autoconf-gmake build system:evans:roberts
draco-test-library:draco test problem library:gray:
draco-linear-algebra:draco linear algebra library:roberts:
\end{verbatim}

\subsection{Responsible Parties}
Listed below are the responsible parties that \gnats\ currently recognizes.
These can be changed by the \gnats\ administrator.

\begin{verbatim}
Responsible
#######################################################################
#
# RTT Responsible Parties
#
mcghee:John M. McGhee:mcghee@lanl.gov
evans:Tom M. Evans:tme@lanl.gov
roberts:Randy M. Roberts:rsqrd@lanl.gov
pautz:Shawn D. Pautz:pautz@lanl.gov
warsa:Jim S. Warsa:warsa@lanl.gov
morel:Jim E. Morel:morel@lanl.gov
furnish:Geoff M. Furnish:furnish@lanl.gov
urbatsch:Todd J. Urbatsch:tmonster@lanl.gov
wareing:Todd A. Wareing:wareing@lanl.gov
archuleta:Denise G. Archuleta:dga@lanl.gov
gray:Mark G. Gray:gray@lanl.gov
lowrie:Robert B. Lowrie:lowrie@lanl.gov
hawkins:William D. Hawkins:dhawkins@lanl.gov
nystrom:William D. Nystrom:wdn@lanl.gov
\end{verbatim}

\section{Conclusion}

This bug tracking system will only work if people use it. I
would encourage all to submit bugs and to track them. Use of the
\gnats\ database does not have to be limited strictly to software bugs but
can also be used to track deficiencies and requests of all
kinds. Such might include: documentation bugs, requests for a changes 
in behavior, requests for additional features, things-to-do, 
and support problems or questions.
 
\closing
\end{document}

%%---------------------------------------------------------------------------%%
%% end of draco-gnats.tex
%%---------------------------------------------------------------------------%%
