%%---------------------------------------------------------------------------%%
%% MTConcept.tex
%% Shawn Pautz
%% Time-stamp: <99/02/09 16:12:45 tme>
%%---------------------------------------------------------------------------%%
\documentclass[11pt]{rnote}
\usepackage[centertags]{amsmath}
\usepackage{amssymb,amsthm,graphicx}
\usepackage[mathcal]{euscript}
\usepackage{../tex/tmadd}
\usepackage{../tex/tmath}
\usepackage{cite}
\usepackage{tabularx}
\usepackage{../tex/stl}
\usepackage{../tex/c++}

%%---------------------------------------------------------------------------%%
%% DEFINE SPECIFIC ENVIRONMENTS HERE
%%---------------------------------------------------------------------------%%
%\newcommand{\elfit}{\ensuremath{\operatorname{Im}(-1/\epsilon(\vq,\omega)}}
%\msection{}-->section commands
%\tradem{}  -->add TM subscript to entry
%\ucatm{}   -->add trademark footnote about entry

%%---------------------------------------------------------------------------%%
%% BEGIN DOCUMENT
%%---------------------------------------------------------------------------%%
\begin{document}

%%---------------------------------------------------------------------------%%
%% OPTIONS FOR NOTE
%%---------------------------------------------------------------------------%%

\toms{Distribution}
%\toms{Joe Sixpak/XTM, MS B226}
\refno{XTM:SDP-99-??? (U)}
\subject{Specification of the \concept{MT} Concept}

%-------NO CHANGES
\divisionname{Applied Theoretical \& Computational Physics Div.}
\groupname{X-TM:Transport Methods Group}
\fromms{Shawn D. Pautz/XTM D409 \\ Randy M. Roberts}
\phone{(505)667--9138}
\originator{sdp}
\typist{sdp}
\date{\today}
%-------NO CHANGES

%-------OPTIONS
%\reference{NPB Star Reimbursable Project}
%\thru{P. D. Soran, XTM, MS B226}
%\enc{list}      
%\attachments{list}
%\cy{list}
%\encas
%\attachmentas
%\attachmentsas 
%-------OPTIONS

%%---------------------------------------------------------------------------%%
%% DISTRIBUTION LIST
%%---------------------------------------------------------------------------%%

\distribution {
  XTM MS D409:\\ 
  J.E. Morel, XTM MS D409\\ 
  G.L. Olson, XTM MS D409\\ 
  J.M. McGhee, XTM MS D409\\ 
  H.G. Hughes, XTM MS D409\\ 
  T.M. Evans, XTM MS D409\\ 
  M.G. Gray, XTM MS D409\\ 
  M.L. Hall, XTM MS D409\\ 
  S.D. Pautz, XTM MS D409\\ 
  R.M. Roberts, XTM MS D409\\ 
  S.A. Turner, XTM MS D409\\ 
  T.J. Urbatsch, XTM MS D409\\ 
}

%%---------------------------------------------------------------------------%%
%% BEGIN NOTE
%%---------------------------------------------------------------------------%%

\opening

\begin{abstract}

In this note we describe the \concept{MT} concept, which defines the
requirements for a \C++ class that provides a variety of services for
an orthogonal 3D Cartesian mesh. This concept was originally defined
for use in the Solon radiation transport package, but is intended to
be used by other similar packages as well. The \concept{MT} concept is 
defined in generic terms that correspond with those used in the
Standard Template Library; hence any class that conforms to the
\concept{MT} requirements is a valid model of the \concept{MT} concept
and also may be used with numerous STL components. Since the
\concept{MT} concept could prove useful to other radiation transport
packages, classes that are models of it have been included in the
Draco component library.

\end{abstract}

\section{Introduction}

The recent development of the \C++ Standard Template Library
(STL)~\cite{ansi:cpp} has
provided \C++ developers with many useful and efficient classes and
algorithms to incorporate in their codes. Of arguably greater
importance, however, it has advanced the appreciation and use of
generic programming techniques. Generic programming offers the
possibility of true code reuse, one of the benefits sought by \C++
users but too often unrealized in the past.

During the design of the Solon radiation transport package the need
for a class to supply data and services related to the mesh was
appreciated. The initial development of this mesh class borrowed many
ideas from the STL; more recently it has been updated to reflect the
generic programming standards set forth in~\cite{au99}. In keeping with the
terminology introduced in~\cite{au99}, we have defined the \concept{MT} {\it
  concept}; a concept is the set of requirements and constraints
imposed on a class or set of classes. Any class meeting the
requirements of a given concept is said to be a {\it model} of that
concept; in particular, any class adhering to the \concept{MT}
requirements is a model of \concept{MT}. Since the mesh requirements
of Solon are incorporated in the \concept{MT} concept, any model of
\concept{MT} may be used in Solon with minimal integration effort,
despite potentially radically different implementation details().

The \concept{MT} concept (and the classes that are models of it) is a
part of the larger Draco effort to define and implement \C++ components 
useful for the development of a wide variety of radiation transport
packages. As packages are developed, and as their needs are met by
defining appropriate concepts and implementing models of those
concepts, it is hoped that each successive package will need to define 
and implement fewer new concepts and instead can reuse previously
designed and tested Draco components.

\section{Concepts}

Besides the \concept{MT} concept itself, there are several associated
concepts that together are used to define the \concept{MT}. Some of
these concepts are already defined in~\cite{au99} and will not be restated
here. The new concepts defined below are specified according to the
format in~\cite{au99}.

\subsection{MT}
A model of \concept{MT} is a class that describes an orthogonal 3D
Cartesian mesh and has nested type definitions for a variety of field
types. The mesh is required to have constant thickness (fixed number
of cells) in any given direction.

\subsubsection{Refinement of}
\concept{Equality Comparable} (Austern, page 85).

\subsubsection{Associated Types}

\begin{typelist}
  \typeitem{Size type}{X::size\cu type}{An unsigned integral type that
    can represent the number of cells in the entire mesh.}
  \typeitem{Field constructor type}{X::FieldConstructor}{The argument
    type for constructing one of the \concept{MT} fields.}
  \typeitem{Cell centered field}{template <class T> X::cctf<T>}{A
    model of \concept{Cell Centered MTField}.}
  \typeitem{Face centered discontinuous field}{template <class T>
    X::fcdtf<T>}{A model of \concept{Face Centered Discontinuous
      MTField}.}
  \typeitem{Node centered field}{template <class T> X::nctf<T>}{A
    model of \concept{Node Centered MTField}.}
  \typeitem{Vertex centered field}{template <class T> X::vctf<T>}{A
    model of \concept{Vertex Centered MTField}.}
  \typeitem{Boundary specified field}{template <class T>
    X::bstf<T>}{A model of \concept{Boundary Specified MTField}.}
  \typeitem{Cell centered scalar field}{X::ccsf}{A typedef for
    \comp{X::cctf<double>}.}
  \typeitem{Face centered discontinuous scalar field}{X::fcdsf}{A
    typedef for \comp{X::fcdtf<double>}.}
  \typeitem{Node centered scalar field}{X::ncsf}{A typedef for
    \comp{X::nctf<double>}.}
  \typeitem{Vertex centered scalar field}{X::vcsf}{A typedef for
    \comp{X::vctf<double>}.}
  \typeitem{Boundary specified scalar field}{X::bssf}{A typedef for
    \comp{X::bstf<double>}.}
  \typeitem{Cell centered integer field}{X::ccif}{A typedef for
    \comp{X::cctf<int>}.}
  \typeitem{Face centered discontinuous integer field}{X::fcdif}{A
    typedef for \comp{X::fcdtf<int>}.}
  \typeitem{Boundary specified integer field}{X::bsif}{A typedef for
    \comp{X::bstf<int>}.}
  \typeitem{Face centered discontinuous vector scalar
    field}{X::fcdvsf}{A typedef for \comp{X::fcdtf<vec>}, where
    \comp{vec} is a model of \concept{DoubleVec}.}
  \typeitem{Node centered vector scalar field}{X::ncvsf}{A typedef for
    \comp{X::nctf<vec>}, where \comp{vec} is a model of
    \concept{DoubleVec}.}
  \typeitem{Vertex centered vector scalar field}{X::vcvsf}{A typedef
    for \comp{X::vctf<vec>}, where \comp{vec} is a model of
    \concept{DoubleVec}.}
  \typeitem{Assign functor}{X::OpAssign}{A model of \concept{Assign
      Op}.}
  \typeitem{Assign with addition functor}{X::OpAddAssign}{A model of
    \concept{Add Assign Op}.}
  \typeitem{Assign with subtraction functor}{X::OpSubAssign}{A model
    of \concept{Sub Assign Op}.}
  \typeitem{Assign with multiplication functor}{X::OpMultAssign}{A
    model of \concept{Mult Assign Op}.}
  \typeitem{Assign with minimum functor}{X::OpMinAssign}{A model of
    \concept{Min Assign Op}.}
  \typeitem{Assign with maximum functor}{X::OpMaxAssign}{A model of
    \concept{Max Assign Op}.}
\end{typelist}

\subsubsection{Notation}
\begin{tabularx}{\linewidth}{>{\setlength{\hsize}{.4\hsize}}X
    >{\setlength{\hsize}{1.6\hsize}}X}
  \comp{X} & A type that is a model of \concept{MT}. \\
  \comp{a,b} & Object of type \comp{X}. \\
  \comp{c} & Object of type \comp{X::ccsf}. \\
  \comp{f} & Object of type \comp{X::fcdsf}. \\
  \comp{fv} & Object of type \comp{X::fcdvsf}. \\
  \comp{v} & Object of type \comp{X::vcsf}. \\
  \comp{n} & Object of type \comp{X::ncsf}. \\
  \comp{ft,ft1,ft2} & Object of type \comp{X::fcdtf<T>}. \\
  \comp{ct} & Object of type \comp{X::cctf<T>}. \\
  \comp{vt} & Object of type \comp{X::vctf<T>}. \\
  \comp{nt} & Object of type \comp{X::nctf<T>}. \\
  \comp{bt} & Object of type \comp{X::bstf<T>}. \\
  \comp{Op} & A type that is a model of \concept{Binary Op}. \\
  \comp{op} & Object of type \comp{Op}. \\
\end{tabularx}

\subsubsection{Definitions}

\begin{deflist}
  \defitem{A {\it cell} of an \concept{MT} is the basic mesh
    element. It occupies a fixed region of space; the union of all
    cells as well as their relation to each other makes up the mesh.}
  \defitem{The {\it center} of a cell is the location of its geometric 
    center.}
  \defitem{A {\it face} of a cell is one of the six rectangular areas
    that define its boundary. We sometimes refer to the face {\it
    center}, which is the location of the geometric center of the
    face. If there are two cells sharing the face, then for the
    purposes of defining a face centered discontinuous field we define 
    a separate face center for each cell.}
  \defitem{A {\it node} of a cell is one of the points where three of
    its faces touch. If more than one cell has such a point at the
    same geometric location, then those cells share the node.}
  \defitem{A {\it vertex} of a cell is located at the same point as
    one of the cell's nodes. Each cell that shares the node has its
    own separately defined vertex.}
  \defitem{A {\it boundary face} of an \concept{MT} is any cell face
    that is located on the boundary of the mesh, i.e. for which there
    is only one cell that is partially defined by it.}
\end{deflist}

\subsubsection{Valid Expressions}

\begin{exprlist}
  \expritem{Equality}{a == b}
    {\begin{tabularx}{\linewidth}{>{\setlength{\hsize}{.5\hsize}}X
    >{\setlength{\hsize}{1.6\hsize}}X}
     Return type: & Convertible to \comp{bool}. \\
     Semantics: & Returns \comp{true} if \comp{\&a == \&b}. Otherwise
     returns \comp{false}. \\
     \end{tabularx}}
  \expritem{Inequality}{a != b}
    {\begin{tabularx}{\linewidth}{>{\setlength{\hsize}{.5\hsize}}X
    >{\setlength{\hsize}{1.6\hsize}}X}
     Return type: & Convertible to \comp{bool}. \\
     Semantics: & Equivalent to \comp{!(a == b)}. \\
     \end{tabularx}}
  \expritem{Number of cells}{a.get\cu ncells()}
    {\begin{tabularx}{\linewidth}{>{\setlength{\hsize}{.5\hsize}}X
    >{\setlength{\hsize}{1.6\hsize}}X}
     Return type: & \comp{size\cu type} \\
     Semantics: & Returns the number of mesh cells in the current
     process. \\
     Postcondition: & \comp{0 <= a.get\cu ncells() <= a.get\cu
       total\cu ncells()}. \\
     \end{tabularx}}
  \expritem{Total number of cells}{a.get\cu total\cu ncells()}
    {\begin{tabularx}{\linewidth}{>{\setlength{\hsize}{.5\hsize}}X
    >{\setlength{\hsize}{1.6\hsize}}X}
     Return type: & \comp{size\cu type} \\
     Semantics: & Returns the total number of mesh cells in all
     processes. \\
     Postcondition: & \comp{0 <= a.get\cu ncells() <= a.get\cu
       total\cu ncells()}. \\
     \end{tabularx}}
  \expritem{$\Delta$x}{a.get\cu dx(c)}
    {\begin{tabularx}{\linewidth}{>{\setlength{\hsize}{.5\hsize}}X
    >{\setlength{\hsize}{1.6\hsize}}X}
     Type requirements: & \comp{c} is mutable. \\
     Return type: & \comp{void} \\
     Precondition: & \comp{c.get\cu Mesh() == a}. \\
     Semantics: & Each cell value in \comp{c} is assigned the value of
     $\Delta$x for that cell. \\
     \end{tabularx}}
  \expritem{$\Delta$y}{a.get\cu dy(c)}
    {\begin{tabularx}{\linewidth}{>{\setlength{\hsize}{.5\hsize}}X
    >{\setlength{\hsize}{1.6\hsize}}X}
     Type requirements: & \comp{c} is mutable. \\
     Return type: & \comp{void} \\
     Precondition: & \comp{c.get\cu Mesh() == a}. \\
     Semantics: & Each cell value in \comp{c} is assigned the value of
     $\Delta$y for that cell. \\
     \end{tabularx}}
  \expritem{$\Delta$z}{a.get\cu dz(c)}
    {\begin{tabularx}{\linewidth}{>{\setlength{\hsize}{.5\hsize}}X
    >{\setlength{\hsize}{1.6\hsize}}X}
     Type requirements: & \comp{c} is mutable. \\
     Return type: & \comp{void} \\
     Precondition: & \comp{c.get\cu Mesh() == a}. \\
     Semantics: & Each cell value in \comp{c} is assigned the value of
     $\Delta$z for that cell. \\
     \end{tabularx}}
  \expritem{X coordinates of cell centers}{a.get\cu xloc(c)}
    {\begin{tabularx}{\linewidth}{>{\setlength{\hsize}{.5\hsize}}X
    >{\setlength{\hsize}{1.6\hsize}}X}
     Type requirements: & \comp{c} is mutable. \\
     Return type: & \comp{void} \\
     Precondition: & \comp{c.get\cu Mesh() == a}. \\
     Semantics: & Each cell value in \comp{c} is assigned the value of
     the x coordinate of that cell's center. \\
     \end{tabularx}}
  \expritem{Y coordinates of cell centers}{a.get\cu yloc(c)}
    {\begin{tabularx}{\linewidth}{>{\setlength{\hsize}{.5\hsize}}X
    >{\setlength{\hsize}{1.6\hsize}}X}
     Type requirements: & \comp{c} is mutable. \\
     Return type: & \comp{void} \\
     Precondition: & \comp{c.get\cu Mesh() == a}. \\
     Semantics: & Each cell value in \comp{c} is assigned the value of
     the y coordinate of that cell's center. \\
     \end{tabularx}}
  \expritem{Z coordinates of cell centers}{a.get\cu zloc(c)}
    {\begin{tabularx}{\linewidth}{>{\setlength{\hsize}{.5\hsize}}X
    >{\setlength{\hsize}{1.6\hsize}}X}
     Type requirements: & \comp{c} is mutable. \\
     Return type: & \comp{void} \\
     Precondition: & \comp{c.get\cu Mesh() == a}. \\
     Semantics: & Each cell value in \comp{c} is assigned the value of
     the z coordinate of that cell's center. \\
     \end{tabularx}}
  \expritem{X coordinates of cell faces}{a.get\cu xloc(f)}
    {\begin{tabularx}{\linewidth}{>{\setlength{\hsize}{.5\hsize}}X
    >{\setlength{\hsize}{1.6\hsize}}X}
     Type requirements: & \comp{f} is mutable. \\
     Return type: & \comp{void} \\
     Precondition: & \comp{f.get\cu Mesh() == a}. \\
     Semantics: & Each face value in \comp{f} is assigned the value of
     the x coordinate of that face's center. \\
     \end{tabularx}}
  \expritem{Y coordinates of cell faces}{a.get\cu yloc(f)}
    {\begin{tabularx}{\linewidth}{>{\setlength{\hsize}{.5\hsize}}X
    >{\setlength{\hsize}{1.6\hsize}}X}
     Type requirements: & \comp{f} is mutable. \\
     Return type: & \comp{void} \\
     Precondition: & \comp{f.get\cu Mesh() == a}. \\
     Semantics: & Each face value in \comp{f} is assigned the value of
     the y coordinate of that face's center. \\
     \end{tabularx}}
  \expritem{Z coordinates of cell faces}{a.get\cu zloc(f)}
    {\begin{tabularx}{\linewidth}{>{\setlength{\hsize}{.5\hsize}}X
    >{\setlength{\hsize}{1.6\hsize}}X}
     Type requirements: & \comp{f} is mutable. \\
     Return type: & \comp{void} \\
     Precondition: & \comp{f.get\cu Mesh() == a}. \\
     Semantics: & Each face value in \comp{f} is assigned the value of
     the z coordinate of that face's center. \\
     \end{tabularx}}
  \expritem{Face normals}{a.get\cu face\cu normals(fv)}
    {\begin{tabularx}{\linewidth}{>{\setlength{\hsize}{.5\hsize}}X
    >{\setlength{\hsize}{1.6\hsize}}X}
     Type requirements: & \comp{fv} is mutable. \\
     Return type: & \comp{void} \\
     Precondition: & \comp{fv.get\cu Mesh() == a}. \\
     Semantics: & Each face value in \comp{fv} is assigned the value of
     the outward unit normal on that face. \\
     \end{tabularx}}
  \expritem{Face areas}{a.get\cu face\cu areas(f)}
    {\begin{tabularx}{\linewidth}{>{\setlength{\hsize}{.5\hsize}}X
    >{\setlength{\hsize}{1.6\hsize}}X}
     Type requirements: & \comp{f} is mutable. \\
     Return type: & \comp{void} \\
     Precondition: & \comp{f.get\cu Mesh() == a}. \\
     Semantics: & Each face value in \comp{f} is assigned the value of
     the area of that face. \\
     \end{tabularx}}
  \expritem{Face lengths}{a.get\cu face\cu lengths(f)}
    {\begin{tabularx}{\linewidth}{>{\setlength{\hsize}{.5\hsize}}X
    >{\setlength{\hsize}{1.6\hsize}}X}
     Type requirements: & \comp{f} is mutable. \\
     Return type: & \comp{void} \\
     Precondition: & \comp{f.get\cu Mesh() == a}. \\
     Semantics: & Each face value in \comp{f} is assigned the value of
     the length of that face, defined as the volume of the cell in
     which the face is located divided by the face area. \\
     \end{tabularx}}
  \expritem{Cell volumes}{a.get\cu cell\cu volumes(c)}
    {\begin{tabularx}{\linewidth}{>{\setlength{\hsize}{.5\hsize}}X
    >{\setlength{\hsize}{1.6\hsize}}X}
     Type requirements: & \comp{c} is mutable. \\
     Return type: & \comp{void} \\
     Precondition: & \comp{c.get\cu Mesh() == a}. \\
     Semantics: & Each cell value in \comp{c} is assigned the value of
     the volume of the cell. \\
     \end{tabularx}}
  \expritem{Vertex volumes}{a.get\cu vertex\cu volumes(v)}
    {\begin{tabularx}{\linewidth}{>{\setlength{\hsize}{.5\hsize}}X
    >{\setlength{\hsize}{1.6\hsize}}X}
     Type requirements: & \comp{v} is mutable. \\
     Return type: & \comp{void} \\
     Precondition: & \comp{v.get\cu Mesh() == a}. \\
     Semantics: & Each vertex value in \comp{v} is assigned the value of
     the volume of the vertex, defined as the volume of the cell in
     which the vertex is located / 8. \\
     \end{tabularx}}
  \expritem{Node volumes}{a.get\cu node\cu volumes(n)}
    {\begin{tabularx}{\linewidth}{>{\setlength{\hsize}{.5\hsize}}X
    >{\setlength{\hsize}{1.6\hsize}}X}
     Type requirements: & \comp{n} is mutable. \\
     Return type: & \comp{void} \\
     Precondition: & \comp{n.get\cu Mesh() == a}. \\
     Semantics: & Each node value in \comp{n} is assigned the value of
     the volume of the node, defined as the sum of the volumes of the
     vertices associated with that node. \\
     \end{tabularx}}
  \expritem{Cell centered to face centered scatter}
    {X::scatter(ft, ct, op)}
    {\begin{tabularx}{\linewidth}{>{\setlength{\hsize}{.5\hsize}}X
    >{\setlength{\hsize}{1.6\hsize}}X}
     Type requirements: & \comp{ft} is mutable. \\
     Return type: & \comp{void} \\
     Preconditions: & \comp{ft.get\cu Mesh() == ct.get\cu Mesh()}. \\
       & \comp{op()} is a valid expression between \comp{ft}'s value
       type and \comp{ct}'s value type. \\
     Semantics: & For each face value of \comp{ft}, \comp{op()} is
     applied in series to the initial face value in \comp{ft} and to
     each cell value in \comp{ct} located in a cell that is adjacent
     to the face. The result is assigned back to the face in
     \comp{ft}. \\
     \end{tabularx}}
  \expritem{Face centered to cell centered scatter}
    {X::scatter(ct, ft, op)}
    {\begin{tabularx}{\linewidth}{>{\setlength{\hsize}{.5\hsize}}X
    >{\setlength{\hsize}{1.6\hsize}}X}
     Type requirements: & \comp{ct} is mutable. \\
     Return type: & \comp{void} \\
     Preconditions: & \comp{ct.get\cu Mesh() == ft.get\cu Mesh()}. \\
       & \comp{op()} is a valid expression between \comp{ct}'s value
       type and \comp{ft}'s value type. \\
     Semantics: & For each cell value of \comp{ct}, \comp{op()} is
     applied in series to the initial cell value in \comp{ct} and to
     each face value in \comp{ft} located in the same cell. The result
     is assigned back to the cell in \comp{ct}. \\
     \end{tabularx}}
  \expritem{Vertex centered to face centered scatter}
    {X::scatter(ft, vt, op)}
    {\begin{tabularx}{\linewidth}{>{\setlength{\hsize}{.5\hsize}}X
    >{\setlength{\hsize}{1.6\hsize}}X}
     Type requirements: & \comp{ft} is mutable. \\
     Return type: & \comp{void} \\
     Preconditions: & \comp{ft.get\cu Mesh() == vt.get\cu Mesh()}. \\
       & \comp{op()} is a valid expression between \comp{ft}'s value
       type and \comp{vt}'s value type. \\
     Semantics: & For each face value of \comp{ft}, \comp{op()} is
     applied in series to the initial face value in \comp{ft} and to
     each vertex value in \comp{vt} located in the same cell and
     adjacent to the face. The result is assigned back to the face in
     \comp{ft}. \\
     \end{tabularx}}
  \expritem{Face centered to vertex centered scatter}
    {X::scatter(vt, ft, op)}
    {\begin{tabularx}{\linewidth}{>{\setlength{\hsize}{.5\hsize}}X
    >{\setlength{\hsize}{1.6\hsize}}X}
     Type requirements: & \comp{vt} is mutable. \\
     Return type: & \comp{void} \\
     Preconditions: & \comp{vt.get\cu Mesh() == ft.get\cu Mesh()}. \\
       & \comp{op()} is a valid expression between \comp{vt}'s value
       type and \comp{ft}'s value type. \\
     Semantics: & For each vertex value of \comp{vt}, \comp{op()} is
     applied in series to the initial vertex value in \comp{vt} and to
     each face value in \comp{ft} located in the same cell and
     adjacent to the vertex. The result is assigned back to the vertex
     in \comp{vt}. \\
     \end{tabularx}}
  \expritem{Vertex centered to node centered scatter}
    {X::scatter(nt, vt, op)}
    {\begin{tabularx}{\linewidth}{>{\setlength{\hsize}{.5\hsize}}X
    >{\setlength{\hsize}{1.6\hsize}}X}
     Type requirements: & \comp{nt} is mutable. \\
     Return type: & \comp{void} \\
     Preconditions: & \comp{nt.get\cu Mesh() == vt.get\cu Mesh()}. \\
       & \comp{op()} is a valid expression between \comp{nt}'s value
       type and \comp{vt}'s value type. \\
     Semantics: & For each node value of \comp{nt}, \comp{op()} is
     applied in series to the initial node value in \comp{nt} and to
     each vertex value in \comp{vt} adjacent to the node. The result
     is assigned back to the node in \comp{nt}. \\
     \end{tabularx}}
  \expritem{Vertex centered to cell centered scatter}
    {X::scatter(ct, vt, op)}
    {\begin{tabularx}{\linewidth}{>{\setlength{\hsize}{.5\hsize}}X
    >{\setlength{\hsize}{1.6\hsize}}X}
     Type requirements: & \comp{ct} is mutable. \\
     Return type: & \comp{void} \\
     Preconditions: & \comp{ct.get\cu Mesh() == vt.get\cu Mesh()}. \\
       & \comp{op()} is a valid expression between \comp{ct}'s value
       type and \comp{vt}'s value type. \\
     Semantics: & For each cell value of \comp{ct}, \comp{op()} is
     applied in series to the initial cell value in \comp{ct} and to
     each vertex value in \comp{vt} in the same cell. The result is
     assigned back to the cell in \comp{ct}. \\
     \end{tabularx}}
  \expritem{Cell centered to face centered gather}
    {X::gather(ft, ct, op)}
    {\begin{tabularx}{\linewidth}{>{\setlength{\hsize}{.5\hsize}}X
    >{\setlength{\hsize}{1.6\hsize}}X}
     Type requirements: & \comp{ft} is mutable. \\
     Return type: & \comp{void} \\
     Preconditions: & \comp{ft.get\cu Mesh() == ct.get\cu Mesh()}. \\
       & \comp{op()} is a valid expression between \comp{ft}'s value
       type and \comp{ct}'s value type. \\
     Semantics: & For each face value of \comp{ft}, \comp{op()} is
     applied between the initial face value in \comp{ft} and to the
     cell value in \comp{ct} in the same cell. The result is assigned
     back to the face in \comp{ft}. \\
     \end{tabularx}}
  \expritem{Face centered to boundary specified gather}
    {X::gather(bt, ft, op)}
    {\begin{tabularx}{\linewidth}{>{\setlength{\hsize}{.5\hsize}}X
    >{\setlength{\hsize}{1.6\hsize}}X}
     Type requirements: & \comp{bt} is mutable. \\
     Return type: & \comp{void} \\
     Preconditions: & \comp{bt.get\cu Mesh() == ft.get\cu Mesh()}. \\
       & \comp{op()} is a valid expression between \comp{bt}'s value
       type and \comp{ft}'s value type. \\
     Semantics: & For each boundary value of \comp{bt}, \comp{op()} is
     applied between the initial boundary value in \comp{bt} and to
     the face value in \comp{ft} adjacent to it. The result is
     assigned back to the boundary in \comp{bt}. \\
     \end{tabularx}}
  \expritem{Boundary specified to face centered gather}
    {X::gather(ft, bt, op)}
    {\begin{tabularx}{\linewidth}{>{\setlength{\hsize}{.5\hsize}}X
    >{\setlength{\hsize}{1.6\hsize}}X}
     Type requirements: & \comp{ft} is mutable. \\
     Return type: & \comp{void} \\
     Preconditions: & \comp{ft.get\cu Mesh() == bt.get\cu Mesh()}. \\
       & \comp{op()} is a valid expression between \comp{ft}'s value
       type and \comp{bt}'s value type. \\
     Semantics: & For each face value of \comp{ft}, \comp{op()} is
     applied between the initial face value in \comp{ft} and to the
     boundary value in \comp{bt} adjacent to it. The result is
     assigned back to the face in \comp{ft}. \\
     \end{tabularx}}
  \expritem{Node centered to vertex centered gather}
    {X::gather(vt, nt, op)}
    {\begin{tabularx}{\linewidth}{>{\setlength{\hsize}{.5\hsize}}X
    >{\setlength{\hsize}{1.6\hsize}}X}
     Type requirements: & \comp{vt} is mutable. \\
     Return type: & \comp{void} \\
     Preconditions: & \comp{vt.get\cu Mesh() == nt.get\cu Mesh()}. \\
       & \comp{op()} is a valid expression between \comp{vt}'s value
       type and \comp{nt}'s value type. \\
     Semantics: & For each vertex value of \comp{vt}, \comp{op()} is
     applied between the initial vertex value in \comp{vt} and to the
     node value in \comp{nt} adjacent to it. The result is assigned
     back to the vertex in \comp{vt}. \\
     \end{tabularx}}
  \expritem{Cell centered to vertex centered gather}
    {X::gather(vt, ct, op)}
    {\begin{tabularx}{\linewidth}{>{\setlength{\hsize}{.5\hsize}}X
    >{\setlength{\hsize}{1.6\hsize}}X}
     Type requirements: & \comp{vt} is mutable. \\
     Return type: & \comp{void} \\
     Preconditions: & \comp{vt.get\cu Mesh() == ct.get\cu Mesh()}. \\
       & \comp{op()} is a valid expression between \comp{vt}'s value
       type and \comp{ct}'s value type. \\
     Semantics: & For each vertex value of \comp{vt}, \comp{op()} is
     applied between the initial vertex value in \comp{vt} and to the
     cell value in \comp{ct} in the same cell. The result is assigned
     back to the vertex in \comp{vt}. \\
     \end{tabularx}}
  \expritem{Face swaps}
    {X::swap\cu faces(ft1, ft2)}
    {\begin{tabularx}{\linewidth}{>{\setlength{\hsize}{.5\hsize}}X
    >{\setlength{\hsize}{1.6\hsize}}X}
     Type requirements: & \comp{ft1} is mutable. \\
     Return type: & \comp{bool} \\
     Preconditions: & \comp{ft1.get\cu Mesh() == ft2.get\cu
       Mesh()}. \\
     Semantics: & For each pair of face values in \comp{ft2}, the
     order of the values is reversed and assigned to the corresponding
     pair in \comp{ft1}. At boundaries (where there are no face pairs)
     the single face value should be assigned \comp{0}. \\
     \end{tabularx}}
  \expritem{Cell centered sum}
    {X::sum(ct)}
    {\begin{tabularx}{\linewidth}{>{\setlength{\hsize}{.5\hsize}}X
    >{\setlength{\hsize}{1.6\hsize}}X}
     Return type: & The value type of \comp{ct}. \\
     Semantics: & The value returned is the sum of all values
     contained in \comp{ct} in all processes. \\
     \end{tabularx}}
  \expritem{Face centered sum}
    {X::sum(ft)}
    {\begin{tabularx}{\linewidth}{>{\setlength{\hsize}{.5\hsize}}X
    >{\setlength{\hsize}{1.6\hsize}}X}
     Return type: & The value type of \comp{ft}. \\
     Semantics: & The value returned is the sum of all values
     contained in \comp{ft} in all processes. \\
     \end{tabularx}}
  \expritem{Boundary specified sum}
    {X::sum(bt)}
    {\begin{tabularx}{\linewidth}{>{\setlength{\hsize}{.5\hsize}}X
    >{\setlength{\hsize}{1.6\hsize}}X}
     Return type: & The value type of \comp{bt}. \\
     Semantics: & The value returned is the sum of all values
     contained in \comp{bt} in all processes. \\
     \end{tabularx}}
\end{exprlist}

\subsubsection{Complexity Guarantees}

\begin{complist}
  \compitem{The comparison operations are amortized constant time.}
  \compitem{All of the other operations defined above are linear in
    the number of cells in the mesh.}
\end{complist}

\subsubsection{Models}

The \comp{Mesh\cu XYZ} class is a model of \concept{MT}.

\subsection{Binary Op}

A \concept{Binary Op} is a \concept{Binary Function} that may alter
the value of its first argument but otherwise does not return a
value.

\subsubsection{Refinement of}
\concept{Binary Function} (Austern, page 112).

\subsubsection{Associated Types}

No additional types beyond those defined in \concept{Binary Function}, 
but the requirements are strengthened. The first argument type must be 
mutable and the result type must be \comp{void}.

\subsubsection{Models}

Any function that takes two arguments, the first of which is mutable,
and has a return type of \comp{void} behaves as a \concept{Binary
  Op}. For example, a function pointer of type \comp{void (*)(double
  \&, const double \&)} is a model of \concept{Binary Op}.

\subsection{Default Constructible Binary Op}

A \concept{Default Constructible Binary Op} is a \concept{Binary Op}
that has a default constructor.

\subsubsection{Refinement of}
\concept{Default Constructible} (Austern, page 84), \concept{Binary
  Op}.

\subsubsection{Models}

All of the \comp{Mesh\cu XYZ} \concept{Binary Op}'s in Draco are
models of \concept{Default Constructible Binary Op}.

\subsection{Assign Op}

An \concept{Assign Op} is a \concept{Default Constructible Binary Op}
that does assignment operations.

\subsubsection{Refinement of}
\concept{Default Constructible Binary Op}.

\subsubsection{Notation}
\begin{tabularx}{\linewidth}{>{\setlength{\hsize}{.4\hsize}}X
    >{\setlength{\hsize}{1.6\hsize}}X}
  \comp{Op} & A type that is a model of \concept{Assign Op}. \\
  \comp{X} & \comp{Op}'s first argument type. \\
  \comp{Y} & \comp{Op}'s second argument type. \\
  \comp{op} & Object of type \comp{Op}. \\
  \comp{x} & Object of type \comp{X}. \\
  \comp{y} & Object of type \comp{Y}. \\
\end{tabularx}

\subsubsection{Valid Expressions}

\begin{exprlist}
  \expritem{Function call}{op(x,y)}
    {\begin{tabularx}{\linewidth}{>{\setlength{\hsize}{.5\hsize}}X
    >{\setlength{\hsize}{1.6\hsize}}X}
     Type requirements: & \comp{x} must be mutable. \\
                        & \comp{operator=} must be defined between
     objects of type \comp{X} and objects of type \comp{Y}. \\
     Return type: & \comp{void}. \\
     Precondition: & The ordered pair \comp{(x,y)} is in the domain of 
     \comp{operator=}. \\
     Semantics: & Equivalent to \comp{operator=} applied between the
     two arguments. \\
     Postcondition: & The value of the first argument is in the range
     of \comp{operator=}. \\
     \end{tabularx}}
\end{exprlist}

\subsubsection{Models}

The \comp{Mesh\cu XYZ::OpAssign} class in Draco is a model of
\concept{Assign Op}.

\subsection{Add Assign Op}

An \concept{Add Assign Op} is a \concept{Default Constructible Binary
  Op} that does assignment with addition operations.

\subsubsection{Refinement of}
\concept{Default Constructible Binary Op}.

\subsubsection{Notation}
\begin{tabularx}{\linewidth}{>{\setlength{\hsize}{.4\hsize}}X
    >{\setlength{\hsize}{1.6\hsize}}X}
  \comp{Op} & A type that is a model of \concept{Add Assign Op}. \\
  \comp{X} & \comp{Op}'s first argument type. \\
  \comp{Y} & \comp{Op}'s second argument type. \\
  \comp{op} & Object of type \comp{Op}. \\
  \comp{x} & Object of type \comp{X}. \\
  \comp{y} & Object of type \comp{Y}. \\
\end{tabularx}

\subsubsection{Valid Expressions}

\begin{exprlist}
  \expritem{Function call}{op(x,y)}
    {\begin{tabularx}{\linewidth}{>{\setlength{\hsize}{.5\hsize}}X
    >{\setlength{\hsize}{1.6\hsize}}X}
     Type requirements: & \comp{x} must be mutable. \\
                        & \comp{operator+=} must be defined between
     objects of type \comp{X} and objects of type \comp{Y}. \\
     Return type: & \comp{void}. \\
     Precondition: & The ordered pair \comp{(x,y)} is in the domain of 
     \comp{operator+=}. \\
     Semantics: & Equivalent to \comp{operator+=} applied between the
     two arguments. \\
     Postcondition: & The value of the first argument is in the range
     of \comp{operator+=}. \\
     \end{tabularx}}
\end{exprlist}

\subsubsection{Models}

The \comp{Mesh\cu XYZ::OpAddAssign} class in Draco is a model of
\concept{Add Assign Op}.

\subsection{Sub Assign Op}

A \concept{Sub Assign Op} is a \concept{Default Constructible Binary
  Op} that does assignment with subtraction operations.

\subsubsection{Refinement of}
\concept{Default Constructible Binary Op}.

\subsubsection{Notation}
\begin{tabularx}{\linewidth}{>{\setlength{\hsize}{.4\hsize}}X
    >{\setlength{\hsize}{1.6\hsize}}X}
  \comp{Op} & A type that is a model of \concept{Sub Assign Op}. \\
  \comp{X} & \comp{Op}'s first argument type. \\
  \comp{Y} & \comp{Op}'s second argument type. \\
  \comp{op} & Object of type \comp{Op}. \\
  \comp{x} & Object of type \comp{X}. \\
  \comp{y} & Object of type \comp{Y}. \\
\end{tabularx}

\subsubsection{Valid Expressions}

\begin{exprlist}
  \expritem{Function call}{op(x,y)}
    {\begin{tabularx}{\linewidth}{>{\setlength{\hsize}{.5\hsize}}X
    >{\setlength{\hsize}{1.6\hsize}}X}
     Type requirements: & \comp{x} must be mutable. \\
                        & \comp{operator-=} must be defined between
     objects of type \comp{X} and objects of type \comp{Y}. \\
     Return type: & \comp{void}. \\
     Precondition: & The ordered pair \comp{(x,y)} is in the domain of 
     \comp{operator-=}. \\
     Semantics: & Equivalent to \comp{operator-=} applied between the
     two arguments. \\
     Postcondition: & The value of the first argument is in the range
     of \comp{operator-=}. \\
     \end{tabularx}}
\end{exprlist}

\subsubsection{Models}

The \comp{Mesh\cu XYZ::OpSubAssign} class in Draco is a model of
\concept{Sub Assign Op}.

\subsection{Mult Assign Op}

A \concept{Mult Assign Op} is a \concept{Default Constructible Binary
  Op} that does assignment with multiplication operations.

\subsubsection{Refinement of}
\concept{Default Constructible Binary Op}.

\subsubsection{Notation}
\begin{tabularx}{\linewidth}{>{\setlength{\hsize}{.4\hsize}}X
    >{\setlength{\hsize}{1.6\hsize}}X}
  \comp{Op} & A type that is a model of \concept{Mult Assign Op}. \\
  \comp{X} & \comp{Op}'s first argument type. \\
  \comp{Y} & \comp{Op}'s second argument type. \\
  \comp{op} & Object of type \comp{Op}. \\
  \comp{x} & Object of type \comp{X}. \\
  \comp{y} & Object of type \comp{Y}. \\
\end{tabularx}

\subsubsection{Valid Expressions}

\begin{exprlist}
  \expritem{Function call}{op(x,y)}
    {\begin{tabularx}{\linewidth}{>{\setlength{\hsize}{.5\hsize}}X
    >{\setlength{\hsize}{1.6\hsize}}X}
     Type requirements: & \comp{x} must be mutable. \\
                        & \comp{operator*=} must be defined between
     objects of type \comp{X} and objects of type \comp{Y}. \\
     Return type: & \comp{void}. \\
     Precondition: & The ordered pair \comp{(x,y)} is in the domain of 
     \comp{operator*=}. \\
     Semantics: & Equivalent to \comp{operator*=} applied between the
     two arguments. \\
     Postcondition: & The value of the first argument is in the range
     of \comp{operator*=}. \\
     \end{tabularx}}
\end{exprlist}

\subsubsection{Models}

The \comp{Mesh\cu XYZ::OpMultAssign} class in Draco is a model of
\concept{Mult Assign Op}.

\subsection{Min Assign Op}

A \concept{Min Assign Op} is a \concept{Default Constructible Binary
  Op} that does assignment with minimum operations.

\subsubsection{Refinement of}
\concept{Default Constructible Binary Op}.

\subsubsection{Notation}
\begin{tabularx}{\linewidth}{>{\setlength{\hsize}{.4\hsize}}X
    >{\setlength{\hsize}{1.6\hsize}}X}
  \comp{Op} & A type that is a model of \concept{Min Assign Op}. \\
  \comp{X} & \comp{Op}'s first argument type. \\
  \comp{Y} & \comp{Op}'s second argument type. \\
  \comp{op} & Object of type \comp{Op}. \\
  \comp{x} & Object of type \comp{X}. \\
  \comp{y} & Object of type \comp{Y}. \\
\end{tabularx}

\subsubsection{Valid Expressions}

\begin{exprlist}
  \expritem{Function call}{op(x,y)}
    {\begin{tabularx}{\linewidth}{>{\setlength{\hsize}{.5\hsize}}X
    >{\setlength{\hsize}{1.6\hsize}}X}
     Type requirements: & \comp{x} must be mutable. \\
                        & \comp{min()} must be defined between
     objects of type \comp{X} and objects of type \comp{Y}. \\
     Return type: & \comp{void}. \\
     Precondition: & The ordered pair \comp{(x,y)} is in the domain of 
     \comp{min()}. \\
     Semantics: & Equivalent to \comp{min()} applied between the
     two arguments. \\
     Postcondition: & The value of the first argument is in the range
     of \comp{min()}. \\
     \end{tabularx}}
\end{exprlist}

\subsubsection{Models}

The \comp{Mesh\cu XYZ::OpMinAssign} class in Draco is a model of
\concept{Min Assign Op}.

\subsection{Max Assign Op}

A \concept{Max Assign Op} is a \concept{Default Constructible Binary
  Op} that does assignment with minimum operations.

\subsubsection{Refinement of}
\concept{Default Constructible Binary Op}.

\subsubsection{Notation}
\begin{tabularx}{\linewidth}{>{\setlength{\hsize}{.4\hsize}}X
    >{\setlength{\hsize}{1.6\hsize}}X}
  \comp{Op} & A type that is a model of \concept{Max Assign Op}. \\
  \comp{X} & \comp{Op}'s first argument type. \\
  \comp{Y} & \comp{Op}'s second argument type. \\
  \comp{op} & Object of type \comp{Op}. \\
  \comp{x} & Object of type \comp{X}. \\
  \comp{y} & Object of type \comp{Y}. \\
\end{tabularx}

\subsubsection{Valid Expressions}

\begin{exprlist}
  \expritem{Function call}{op(x,y)}
    {\begin{tabularx}{\linewidth}{>{\setlength{\hsize}{.5\hsize}}X
    >{\setlength{\hsize}{1.6\hsize}}X}
     Type requirements: & \comp{x} must be mutable. \\
                        & \comp{max()} must be defined between
     objects of type \comp{X} and objects of type \comp{Y}. \\
     Return type: & \comp{void}. \\
     Precondition: & The ordered pair \comp{(x,y)} is in the domain of 
     \comp{max()}. \\
     Semantics: & Equivalent to \comp{max()} applied between the
     two arguments. \\
     Postcondition: & The value of the first argument is in the range
     of \comp{max()}. \\
     \end{tabularx}}
\end{exprlist}

\subsubsection{Models}

The \comp{Mesh\cu XYZ::OpMaxAssign} class in Draco is a model of
\concept{Max Assign Op}.

\subsection{Expression Enabled Container}

An \concept{Expression Enabled Container} is a \concept{Container} for 
which all of the usual expression template expressions have been
defined. These expressions are not necessarily implemented with
expression templates.

\subsubsection{Refinement of}
\concept{Container} (Austern, page 125).

\subsubsection{Associated Types}

The same as for \concept{Container}, except that the requirements for
\comp{X::value\cu type} are strengthened. Each of the valid
expressions listed below must have overloaded versions defined for
\comp{X::value\cu type}.

\subsubsection{Notation}
\begin{tabularx}{\linewidth}{>{\setlength{\hsize}{.4\hsize}}X
    >{\setlength{\hsize}{1.6\hsize}}X}
  \comp{X} & A type that is a model of \concept{Expression Enabled
    Container}. \\
  \comp{T} & \comp{X}'s value type. \\
  \comp{a} & Object of type \comp{X}. \\
  \comp{b,c} & Object of type \comp{X} or type \comp{T} or an
  expression. \\
\end{tabularx}

\subsubsection{Definitions}

\begin{deflist}
  \defitem{An {\it expression} is any combination of the
    non-assignment operations listed below operating on objects of
    type \comp{X} or \comp{T}. The return type of an expression is not 
    specified, but the expression must behave semantically as if the
    return type were \comp{X\&}.}
\end{deflist}

\subsubsection{Valid Expressions}

In addition to the expressions defined in \concept{Container}, the
following expressions must be valid.

\begin{exprlist}
  \expritem{Assignment}{a = b}
    {\begin{tabularx}{\linewidth}{>{\setlength{\hsize}{.5\hsize}}X
    >{\setlength{\hsize}{1.6\hsize}}X}
     Type requirements: & \comp{a} is mutable. \\
     Return type: & \comp{X\&}. \\
     Precondition: & \comp{a.size() == b.size()}, unless \comp{b} is
     an object of type \comp{T}. ??? \\
     Postcondition: & \comp{a.size() == b.size()}, unless \comp{b} is
     an object of type \comp{T}. If \comp{b} is an object of type
     \comp{T}, then each element of \comp{a} is equal to \comp{b}.
     Otherwise \comp{a} contains a copy of each of \comp{b}'s
     elements. \\
     \end{tabularx}}
  \expritem{Assignment with addition}{a += b}
    {\begin{tabularx}{\linewidth}{>{\setlength{\hsize}{.5\hsize}}X
    >{\setlength{\hsize}{1.6\hsize}}X}
     Type requirements: & \comp{a} is mutable. \\
     Return type: & \comp{X\&}. \\
     Precondition: & \comp{a.size() == b.size()}, unless \comp{b} is
     an object of type \comp{T}. ??? \\
     Semantics: & Equivalent to \comp{T::operator+=} being applied to
     each element of \comp{a} with \comp{b}, if \comp{b} is an object
     of type \comp{T}. Otherwise equivalent to \comp{T::operator+=}
     being applied to corresponding elements of \comp{a} and
     \comp{b}. \\
     Postcondition: & \comp{a.size() == b.size()}, unless \comp{b} is
     an object of type \comp{T}. \\
     \end{tabularx}}
  \expritem{Assignment with subtraction}{a -= b}
    {\begin{tabularx}{\linewidth}{>{\setlength{\hsize}{.5\hsize}}X
    >{\setlength{\hsize}{1.6\hsize}}X}
     Type requirements: & \comp{a} is mutable. \\
     Return type: & \comp{X\&}. \\
     Precondition: & \comp{a.size() == b.size()}, unless \comp{b} is
     an object of type \comp{T}. ??? \\
     Semantics: & Equivalent to \comp{T::operator-=} being applied to
     each element of \comp{a} with \comp{b}, if \comp{b} is an object
     of type \comp{T}. Otherwise equivalent to \comp{T::operator-=}
     being applied to corresponding elements of \comp{a} and
     \comp{b}. \\
     Postcondition: & \comp{a.size() == b.size()}, unless \comp{b} is
     an object of type \comp{T}. \\
     \end{tabularx}}
  \expritem{Assignment with multiplication}{a *= b}
    {\begin{tabularx}{\linewidth}{>{\setlength{\hsize}{.5\hsize}}X
    >{\setlength{\hsize}{1.6\hsize}}X}
     Type requirements: & \comp{a} is mutable. \\
     Return type: & \comp{X\&}. \\
     Precondition: & \comp{a.size() == b.size()}, unless \comp{b} is
     an object of type \comp{T}. ??? \\
     Semantics: & Equivalent to \comp{T::operator*=} being applied to
     each element of \comp{a} with \comp{b}, if \comp{b} is an object
     of type \comp{T}. Otherwise equivalent to \comp{T::operator*=}
     being applied to corresponding elements of \comp{a} and
     \comp{b}. \\
     Postcondition: & \comp{a.size() == b.size()}, unless \comp{b} is
     an object of type \comp{T}. \\
     \end{tabularx}}
  \expritem{Assignment with division}{a /= b}
    {\begin{tabularx}{\linewidth}{>{\setlength{\hsize}{.5\hsize}}X
    >{\setlength{\hsize}{1.6\hsize}}X}
     Type requirements: & \comp{a} is mutable. \\
     Return type: & \comp{X\&}. \\
     Precondition: & \comp{a.size() == b.size()}, unless \comp{b} is
     an object of type \comp{T}. ??? \\
     Semantics: & Equivalent to \comp{T::operator/=} being applied to
     each element of \comp{a} with \comp{b}, if \comp{b} is an object
     of type \comp{T}. Otherwise equivalent to \comp{T::operator/=}
     being applied to corresponding elements of \comp{a} and
     \comp{b}. \\
     Postcondition: & \comp{a.size() == b.size()}, unless \comp{b} is
     an object of type \comp{T}. \\
     \end{tabularx}}
  \expritem{Unary plus}{+b}
    {\begin{tabularx}{\linewidth}{>{\setlength{\hsize}{.5\hsize}}X
    >{\setlength{\hsize}{1.6\hsize}}X}
     Return type: & Undefined if \comp{b} is not an object of type
     \comp{T}. Semantically, though, it should behave as if it were
     \comp{X\&}. \\
     Precondition: & Unless \comp{b} is an object of type \comp{T},
     this expression must be used in a statement that contains one of
     the assignment operators. \\
     Postcondition: & Unless \comp{b} is an object of type \comp{T},
     the resulting expression should behave semantically as a
     \comp{X\&} in which each element has been assigned the result of
     \comp{T::operator+} applied to corresponding elements of
     \comp{b}. \\
     \end{tabularx}}
  \expritem{Unary minus}{-b}
    {\begin{tabularx}{\linewidth}{>{\setlength{\hsize}{.5\hsize}}X
    >{\setlength{\hsize}{1.6\hsize}}X}
     Return type: & Undefined if \comp{b} is not an object of type
     \comp{T}. Semantically, though, it should behave as if it were
     \comp{X\&}. \\
     Precondition: & Unless \comp{b} is an object of type \comp{T},
     this expression must be used in a statement that contains one of
     the assignment operators. \\
     Postcondition: & Unless \comp{b} is an object of type \comp{T},
     the resulting expression should behave semantically as a
     \comp{X\&} in which each element has been assigned the result of
     \comp{T::operator-} applied to corresponding elements of
     \comp{b}. \\
     \end{tabularx}}
  \expritem{Addition}{b + c}
    {\begin{tabularx}{\linewidth}{>{\setlength{\hsize}{.5\hsize}}X
    >{\setlength{\hsize}{1.6\hsize}}X}
     Type requirements: & \comp{a} is mutable. \\
     Return type: & Undefined if either \comp{b} or \comp{c} is not an
     object of type \comp{T}. Semantically, though, it should behave
     as if it were \comp{X\&}. \\
     Precondition: & \comp{b.size() == c.size()}, unless either
     \comp{b} or \comp{c} is an object of type \comp{T}. ??? \\
     Semantics: & Equivalent to \comp{T::operator+} being applied to
     each element of \comp{b} (or \comp{b} itself if \comp{b} is an
     object of type \comp{T}) with corresponding elements of \comp{c}
     (or \comp{c} itself if \comp{c} is an object of type
     \comp{T}). \\
     \end{tabularx}}
  \expritem{Subtraction}{b - c}
    {\begin{tabularx}{\linewidth}{>{\setlength{\hsize}{.5\hsize}}X
    >{\setlength{\hsize}{1.6\hsize}}X}
     Type requirements: & \comp{a} is mutable. \\
     Return type: & Undefined if either \comp{b} or \comp{c} is not an
     object of type \comp{T}. Semantically, though, it should behave
     as if it were \comp{X\&}. \\
     Precondition: & \comp{b.size() == c.size()}, unless either
     \comp{b} or \comp{c} is an object of type \comp{T}. ??? \\
     Semantics: & Equivalent to \comp{T::operator-} being applied to
     each element of \comp{b} (or \comp{b} itself if \comp{b} is an
     object of type \comp{T}) with corresponding elements of \comp{c}
     (or \comp{c} itself if \comp{c} is an object of type
     \comp{T}). \\
     \end{tabularx}}
  \expritem{Multiplication}{b * c}
    {\begin{tabularx}{\linewidth}{>{\setlength{\hsize}{.5\hsize}}X
    >{\setlength{\hsize}{1.6\hsize}}X}
     Type requirements: & \comp{a} is mutable. \\
     Return type: & Undefined if either \comp{b} or \comp{c} is not an
     object of type \comp{T}. Semantically, though, it should behave
     as if it were \comp{X\&}. \\
     Precondition: & \comp{b.size() == c.size()}, unless either
     \comp{b} or \comp{c} is an object of type \comp{T}. ??? \\
     Semantics: & Equivalent to \comp{T::operator*} being applied to
     each element of \comp{b} (or \comp{b} itself if \comp{b} is an
     object of type \comp{T}) with corresponding elements of \comp{c}
     (or \comp{c} itself if \comp{c} is an object of type
     \comp{T}). \\
     \end{tabularx}}
  \expritem{Division}{b / c}
    {\begin{tabularx}{\linewidth}{>{\setlength{\hsize}{.5\hsize}}X
    >{\setlength{\hsize}{1.6\hsize}}X}
     Type requirements: & \comp{a} is mutable. \\
     Return type: & Undefined if either \comp{b} or \comp{c} is not an
     object of type \comp{T}. Semantically, though, it should behave
     as if it were \comp{X\&}. \\
     Precondition: & \comp{b.size() == c.size()}, unless either
     \comp{b} or \comp{c} is an object of type \comp{T}. ??? \\
     Semantics: & Equivalent to \comp{T::operator/} being applied to
     each element of \comp{b} (or \comp{b} itself if \comp{b} is an
     object of type \comp{T}) with corresponding elements of \comp{c}
     (or \comp{c} itself if \comp{c} is an object of type
     \comp{T}). \\
     \end{tabularx}}
\end{exprlist}

\subsubsection{Complexity Guarantees}

All of the operations defined above are linear in the container's
size.

\subsubsection{Models}

All of the \comp{Mesh\cu XYZ} fields are models of \concept{Expression
  Enabled Container}.

\subsection{Expression Enabled Forward Container}

An \concept{Expression Enabled Forward Container} is a
\concept{Forward Container} for which all of the usual expression
template expressions have been defined. These expressions are not
necessarily implemented with expression templates.

\subsubsection{Refinement of}
\concept{Forward Container} (Austern, page 131), \concept{Expression
  Enabled Container}.

\subsubsection{Models}

All of the \comp{Mesh\cu XYZ} fields are models of \concept{Expression
  Enabled Forward Container}.

\subsection{Expression Enabled Random Access Container}

An \concept{Expression Enabled Random Access Container} is a
\concept{Random Access Container} for which all of the usual
expression template expressions have been defined. These expressions
are not necessarily implemented with expression templates.

\subsubsection{Refinement of}
\concept{Random Access Container} (Austern, page 135),
\concept{Expression Enabled Forward Container}.

\subsubsection{Models}

The \comp{Mesh\cu XYZ::tiny\cu vec<>} class is a model of
\concept{Expression Enabled Random Access Container}.

\subsection{DoubleVec}

A \concept{DoubleVec} is an \concept{Expression Enabled Random Access
  Container} that stores a 3D vector of \comp{double}s.

\subsubsection{Refinement of}
\concept{Expression Enabled Random Access Container}.

\subsubsection{Associated Types}

No additional types beyond those defined in \concept{Expression
Enabled Random Access Container}, but the requirements for the value
type are strengthened. It must be type \comp{double}.

\subsubsection{Notation}
\begin{tabularx}{\linewidth}{>{\setlength{\hsize}{.4\hsize}}X
    >{\setlength{\hsize}{1.6\hsize}}X}
  \comp{X} & A type that is a model of \concept{DoubleVec}. \\
  \comp{a} & Object of type \comp{X}. \\
\end{tabularx}

\subsubsection{Valid Expressions}

In addition to the expressions defined in \concept{Expression Enabled
  Random Access Container}, the following expressions must be valid.

\begin{exprlist}
  \expritem{Size}{a.size()}
    {\begin{tabularx}{\linewidth}{>{\setlength{\hsize}{.5\hsize}}X
    >{\setlength{\hsize}{1.6\hsize}}X}
     Return type: & \comp{size\cu type} \\
     Semantics: & Returns the size of the container \comp{a}, that is, 
     the number of elements it contains. \\
     Postcondition: & \comp{3 == a.size() == a.max\cu size()}. \\
     \end{tabularx}}
  \expritem{Maximum size}{a.max\cu size()}
    {\begin{tabularx}{\linewidth}{>{\setlength{\hsize}{.5\hsize}}X
    >{\setlength{\hsize}{1.6\hsize}}X}
     Return type: & \comp{size\cu type} \\
     Semantics: & Returns an upper bound on the container's size. \\
     Postcondition: & \comp{3 == a.size() == a.max\cu size()}. \\
     \end{tabularx}}
\end{exprlist}

\subsubsection{Models}

The \comp{Mesh\cu XYZ::tiny\cu vec<double>} class is a model of
\concept{DoubleVec}.

\subsection{MTField}

An \concept{MTField} is an \concept{Expression Enabled Forward
  Container} that stores a data field associated with an
\concept{MT}.

\subsubsection{Refinement of}
\concept{Expression Enabled Forward Container}.

\subsubsection{Associated Types}

The requirements for the iterator types are strengthened. The
iterators must iterate only over those parts of the field that are in
the current process. In addition to the types defined in
\concept{Expression Enabled Forward Container}, the following types
are defined.

\begin{typelist}
  \typeitem{Mesh type}{M}{The type of the mesh object with respect to which 
    an \concept{MTField} is defined.}
  \typeitem{Field constructor type}{M::FieldConstructor}{The argument
    type for constructing an \concept{MTField}.}
\end{typelist}

\subsubsection{Notation}
\begin{tabularx}{\linewidth}{>{\setlength{\hsize}{.4\hsize}}X
    >{\setlength{\hsize}{1.6\hsize}}X}
  \comp{X} & A type that is a model of \concept{MTField}. \\
  \comp{M} & A type that is a model of \concept{MT}. \\
  \comp{a} & Object of type \comp{X}. \\
  \comp{f} & Object of type \comp{M::FieldConstructor}. \\
\end{tabularx}

\subsubsection{Valid Expressions}

In addition to the expressions defined in \concept{Expression Enabled
  Forward Container}, the following expressions must be valid.

\begin{exprlist}
  \expritem{Field Constructor}{X(f)}
    {\begin{tabularx}{\linewidth}{>{\setlength{\hsize}{.5\hsize}}X
    >{\setlength{\hsize}{1.6\hsize}}X}
     Return type: & \comp{X} \\
     \end{tabularx}}
  \expritem{Obtain mesh}{a.get\cu Mesh()}
    {\begin{tabularx}{\linewidth}{>{\setlength{\hsize}{.5\hsize}}X
    >{\setlength{\hsize}{1.6\hsize}}X}
     Return type: & \comp{M\&} \\
     Semantics: & Returns a reference to the mesh object with respect
     to which \comp{a} is defined. \\
     \end{tabularx}}
  \expritem{Obtain field constructor}{a.get\cu FieldConstructor()}
    {\begin{tabularx}{\linewidth}{>{\setlength{\hsize}{.5\hsize}}X
    >{\setlength{\hsize}{1.6\hsize}}X}
     Return type: & \comp{M::FieldConstructor\&} \\
     Postcondition: & The returned reference, when used as an argument 
     to the \concept{MTField} constructor, yields an object defined
     with respect to the same mesh as \comp{a}. \\
     \end{tabularx}}
  \expritem{Beginning of range}{a.begin()}
    {\begin{tabularx}{\linewidth}{>{\setlength{\hsize}{.5\hsize}}X
    >{\setlength{\hsize}{1.6\hsize}}X}
     Return type: & \comp{iterator} if \comp{a} is mutable, otherwise
     \comp{const\cu iterator}. \\
     Semantics: & Returns an iterator pointing to the first element in 
     \comp{a} in the current process. \\
     Postcondition: & \comp{a.begin()} is either dereferenceable or
     past the end. It is past the end if and only if \comp{a.size() == 
       0}. \\
     \end{tabularx}}
  \expritem{End of range}{a.end()}
    {\begin{tabularx}{\linewidth}{>{\setlength{\hsize}{.5\hsize}}X
    >{\setlength{\hsize}{1.6\hsize}}X}
     Return type: & \comp{iterator} if \comp{a} is mutable, otherwise
     \comp{const\cu iterator}. \\
     Semantics: & Returns an iterator pointing to one past the last
     element in \comp{a} in the current process. \\
     Postcondition: & \comp{a.end()} is past the end. The two
     iterators \comp{a.begin()} and \comp{a.end()} if and only if
     \comp{a.size() == 0}. \\
     \end{tabularx}}
  \expritem{Size}{a.size()}
    {\begin{tabularx}{\linewidth}{>{\setlength{\hsize}{.5\hsize}}X
    >{\setlength{\hsize}{1.6\hsize}}X}
     Return type: & \comp{size\cu type} \\
     Semantics: & Returns the size of the field of \comp{a} in the
     current process, that is, the number of elements it contains. \\
     Postcondition: & \comp{0 <= a.size() == a.max\cu size()}. \\
     \end{tabularx}}
  \expritem{Maximum size}{a.max\cu size()}
    {\begin{tabularx}{\linewidth}{>{\setlength{\hsize}{.5\hsize}}X
    >{\setlength{\hsize}{1.6\hsize}}X}
     Return type: & \comp{size\cu type} \\
     Semantics: & Returns an upper bound on the container's size. \\
     Postcondition: & \comp{0 <= a.size() == a.max\cu size()}. \\
     \end{tabularx}}
\end{exprlist}

\subsubsection{Complexity Guarantees}

\begin{complist}
  \compitem{The field constructor is linear in the container's size.}
  \compitem{\comp{get\cu Mesh()} and \comp{get\cu FieldConstructor()}
    are amortized constant time.}
\end{complist}

\subsubsection{Models}

All of the \comp{Mesh\cu XYZ} fields are models of \concept{MTField}.

\subsection{Cell Centered MTField}

A \concept{Cell Centered MTField} is an \concept{MTField} that
contains a value for each cell in the associated mesh.

\subsubsection{Refinement of}
\concept{MTField}.

\subsubsection{Models}

The \comp{Mesh\cu XYZ::cctf<>} class is a model of \concept{Cell
  Centered MTField}.

\subsection{Face Centered Discontinuous MTField}

A \concept{Face Centered Discontinuous MTField} is an
\concept{MTField} that contains a value for each face in the
associated mesh.

\subsubsection{Refinement of}
\concept{MTField}.

\subsubsection{Models}

The \comp{Mesh\cu XYZ::fcdtf<>} class is a model of \concept{Face
  Centered Discontinuous MTField}.

\subsection{Node Centered MTField}

A \concept{Node Centered MTField} is an \concept{MTField} that
contains a value for each node in the associated mesh.

\subsubsection{Refinement of}
\concept{MTField}.

\subsubsection{Models}

The \comp{Mesh\cu XYZ::nctf<>} class is a model of \concept{Node
  Centered MTField}.

\subsection{Vertex Centered MTField}

A \concept{Vertex Centered MTField} is an \concept{MTField} that
contains a value for each vertex in the associated mesh.

\subsubsection{Refinement of}
\concept{MTField}.

\subsubsection{Models}

The \comp{Mesh\cu XYZ::vctf<>} class is a model of \concept{Vertex
  Centered MTField}.

\subsection{Boundary Specified MTField}

A \concept{Boundary Specified MTField} is an \concept{MTField} that
contains a value for each boundary face in the associated mesh.

\subsubsection{Refinement of}
\concept{MTField}.

\subsubsection{Models}

The \comp{Mesh\cu XYZ::bstf<>} class is a model of \concept{Boundary
  Specified  MTField}.

\bibliographystyle{../tex/rnote}
\bibliography{../bib/draco}

\closing
\end{document}

%%---------------------------------------------------------------------------%%
%% end of test.tex
%%---------------------------------------------------------------------------%%
