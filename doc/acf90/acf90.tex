%  ========================================================================  %
% 
% 	Author:	Mark G. Gray
% 		Los Alamos National Laboratory
% 	Date:	Wed Apr  5 09:42:35 MDT 2000
% 
% 	Copyright (c) 2000 U. S. Department of Energy. All rights reserved.
% 
%  	$Id$
% 
%  ========================================================================  %

\documentclass[11pt]{nmemo}
\usepackage[centertags]{amsmath}
\usepackage{amssymb,amsthm,graphicx}
\usepackage[mathcal]{euscript}
\usepackage{tmadd,tmath}
\usepackage{c++}
\usepackage{cite}

%%---------------------------------------------------------------------------%%
%% BEGIN DOCUMENT
%%---------------------------------------------------------------------------%%

\newcommand{\fninety}{\texttt{Fortran~90}}
\newcommand{\cpp}{\texttt{C++}}
\newcommand{\withfninety}{\texttt{--with-f90}}
\newcommand{\langfninety}{\texttt{AC\_LANG\_F90}}
\newcommand{\progfninety}{\texttt{AC\_PROG\_F90}}
\newcommand{\requiregmfour}{\texttt{AC\_REQUIRE\_GM4}}
\newcommand{\proggmfour}{\texttt{AC\_PROG\_GM4}}

\begin{document}

%%---------------------------------------------------------------------------%%
%% OPTIONS FOR NOTE
%%---------------------------------------------------------------------------%%

\toms{Distribution}
\refno{X6:MGG-00-XX (U)}
\subject{\fninety\ Build Support in Draco}
\divisionname{Applied Physics Division}
\groupname{X--6:Transport Methods Group}
\fromms{Mark G. Gray/X--6, MS D409}
\phone{(505)667--5341}
\originator{mgg}
\typist{mgg}
\date{\today}

%%---------------------------------------------------------------------------%%
%% DISTRIBUTION LIST
%%---------------------------------------------------------------------------%%

\distribution {
  J.E. Morel, XTM MS D409\\
  J.M. McGhee, XTM MS D409\\
  H.G. Hughes, XTM MS D409\\
  T.M. Evans, XTM MS D409\\
  M.G. Gray, XTM MS D409\\
  S.D. Pautz, XTM MS D409\\
  R.M. Roberts, XTM MS D409\\
  T.J. Urbatsch, XTM MS D409\\
  T.A. Wareing, XTM MS D409\\
  J.S. Warsa, XTM MS D409\\
  C.J. Gesh, XTM MS D409\\
  W.D. Hawkins, XTM MS D409\\
  B.T. Adams, XTM MS D409 \\
  M.L. Alme, XTM MS D409\\
  J.C. Gulick, XTM D409\\
  K.G. Thompson, XCI, MS F663\\
  R.B. Lowrie, XHM MS D413 
  }

%%---------------------------------------------------------------------------%%
%% BEGIN NOTE
%%---------------------------------------------------------------------------%%

\opening

% I want to tell John that...

I have added rudimentary \fninety\ compiler support to the Draco build
system\cite{draco-build}.  The autoconf\cite{autoconf} macros in
\texttt{draco/config} now support the \withfninety\ argument, which
allows the user to specify the \fninety\ compiler to use; the
\langfninety\ and \progfninety\ macros to specify \fninety\ as the
language used in compiler tests and to test the selected compiler,
respectively; and the \proggmfour\ and \requiregmfour\ macros to find
and require the GNU m4\cite{m4} preprocessor, respectively.  This
argument and these macros provide basic functionality for configuring
a \fninety\ package.  They also provide the framework for full
\fninety\ support in Draco, the template for other language support in
Draco, and suggest a possible refactoring of the existing build
macros.

\section{Background}

Draco was originally concieved as a collection of \cpp\ packages that
could be used to construct transport codes.  An unexpected bonus of
the Draco project was a robust build system, based on autoconf, that
could be used to configure a package of \cpp\ code for various
platforms.  In order to support such reuse, the build system required
much careful planning and hard work to create, but provided a system
that is easy to use.

One feature missing from the build system was support for \fninety.
Although Draco will remain primarily a \cpp repository, \fninety\ 
interface packages, numerical packages, and vendor packages are
planned additions.  Further, just as the build system in Draco is used
in several \cpp\ projects, a common build system which could
be used by several \fninety\ projects is desirable.

I had written rudimentary \fninety\ support as part of the Zathras and
Centauri projects.  This support was based on the autoconf support
written for Dante by Randy Roberts.  With the advent of the new Linux
workstations and their new Fujitsu compiler I needed to at least
revisit my old \fninety\ autoconf macros.  This seemed to be an ideal
time to look into incorporating that \fninety\ support in Draco.

I examined Tom Evans work on the autoconf macros for the build system,
and found the design sound and highly ameanable to the necessary
\fninety\ additions.  This memo documents the addition of several
necessary build system features for \fninety\ support.

\newpage

\section{Introduction}

Configuring for \fninety\ is a complicated business.  First, autoconf
was clearly written primarily to support \texttt{C}, although it does
have fair \cpp\ support.  It was not designed to make supporting other
languages easy.  Next, there is little consistency in \fninety\
compilers across platforms.  The compiler name, flags, default file
extensions, and method of handling modules varies greatly from product
to product.  Finally, this inconsistency makes the autoconf method of
guessing actually dangerous.  In order to add \fninety\ support some
native autoconf must be rewritten, a database of \fninety\ compiler
information must be available, and a better guessing heuristic must be
devised.  In the spirit of the Draco build system, \fninety\ support
aims at simplifying the configuration for the user, at the expense of
considerable work on the part of the build system.

\section{New Arguments}

The \fninety\ compiler can be specified by giving the \withfninety\ 
argument to configure.  The currently supported compilers are shown in
Table~\ref{tbl:compilers}.

\begin{table}[hb]
\begin{center}
\caption{Supported \fninety\ compilers}\label{tbl:compilers}
\begin{tabular}{l|l|l|l}
Compiler        & Executable & Target  & Target \\
(\withfninety=) & File       & OS      & Platform \\ \hline
Fujitsu         & f90        & Linux   & ix86 \\
XL	        & xlf90      & AUX     & RS6000 \\
WorkShop        & f90	     & Solaris & Sparc \\
Cray            & f90        & UNICOS  & Y-MP \\
MIPS            & f90        & IRIX    & SGI  
\end{tabular}
\end{center}
\end{table}
If \withfninety\ is given without any arguments, an attempt to guess the
compiler is made by the configure script based on the target.  This
guess is verified by running the compiler with \texttt{--version} (or
its equivalent) and checking the output for the product name.  The
\withfninety\ argument is ignored if the \langfninety\ macro is not
present in \texttt{configure.in}.

A new compiler can be added by putting a database entry for it in
\texttt{ac\_compiler.m4}, adding it to the case statement in
\texttt{ac\_dracoenv.m4}, and adding it to the argument option string
in \texttt{ac\_dracoarg.m4}.  The \withfninety\ argument is based on
the \texttt{--with-cxx} argument in \texttt{ac\_dracoarg.m4}.

\section{New Macros}

Four new macros have been added for \fninety\ support:

\begin{description}
\item[\langfninety:] sets up the compile and link test invocation
using \texttt{F90} and the extension \texttt{F90EXT}.  This macro is
required in \texttt{configure.in} for \fninety\ usage.  It is based on
the \texttt{AC\_LANG\_C} and \texttt{AC\_LANG\_CPLUSPLUS} macros in
\texttt{acgeneral.m4}.

\item[\proggmfour:] sets the output variable \texttt{GM4} to a command
that runs the GNU m4 preprocessor.  Looks for \texttt{gm4} and then
\texttt{m4}, and verifies the gnu version by running
\texttt{\${GM4}~--version}.  It is based on the \texttt{AC\_PROG\_CPP}
and \texttt{AC\_PROG\_CPPCXX} macros in \texttt{acspecific.m4}.

\item[\requiregmfour:] ensures that \texttt{GM4} has been found by
calling \proggmfour\ if it hasn't already been called.  It is based on
the \texttt{AC\_REQUIRE\_CPP} macro in \texttt{acspecific.m4}.

\item[\progfninety:] determines the \fninety\ compiler to use and the
appropriate extensions for free format source code.  Sets the
following output variables:
\begin{description}
\item[\texttt{F90}:] name of a working \fninety\ compiler.
\item[\texttt{F90FLAGS}:] vendor specific default flags for the
\fninety\ compiler. 
Changes appropriately according to the \texttt{--enable-debug} and
\texttt{--with-opt} configure arguments.
\item[\texttt{F90EXT}:] vendor specific extension of \fninety\ input files.
\item[\texttt{F90FREE}:] vendor specific flag to specify free form
source input. 
\item[\texttt{F90FIXED}:] vendor specific flag to specify fixed form
source input. 
\item[\texttt{MODNAME}:] vendor specific module file name.  Module
files containing interface information are generated by the compiler
when a file containing a module is found.  Module files may be named
after the file containing the module, or after the module name, in
either all caps, all lowercase, or mixed case.  Use a compile test to
determine which is the case.   
\item[\texttt{MODSUFFIX}:] vendor specific module extension name.
Module files may have extension \texttt{mod}, \texttt{M}, \texttt{o}
(i.e. be part of the object file), or something else.  Use a compile
test to determine which is the case.
\item[\texttt{MODFLAG}:] vendor specific module include directory
flag, typically \texttt{-I} or \texttt{-M}.
\end{description}
It is based on the \texttt{AC\_PROG\_CC} and \texttt{AC\_PROG\_CXX}
macros in \texttt{acspecific.m4}.
\end{description}

These macros have only been tested on a limited number of machines.
\progfninety\ can fail due to vendor non-standard file extentions or
incorrect free/fixed source defaults.  Output variables are correctly
set for only a few known targets.  As with other autoconf macros, any
file named [Cc]onftest* will be overwritten!

\section{Example Fortran 90 Configuration}

\begin{figure}[hbt]
\begin{verbatim}
AC_INIT(acf90)                  dnl 1
AC_CONFIG_AUX_DIR(../../config) dnl 2
                                dnl 3
AC_LANG_F90                     dnl 4
AC_DRACO_ENV                    dnl 5
AC_PROG_F90                     dnl 6
AC_PROG_GM4                     dnl 7
                                dnl 8
AC_OUTPUT(Makefile)             dnl 9
\end{verbatim}
\caption{Sample \texttt{configure.in}. Line 1 processes command line
arguments and finds the source file directory.  Line 2 specifies where
to find \texttt{install-sh}, \texttt{config.sub}, and
\texttt{config.guess}.  Line 4 sets the language to \fninety, with
appropriate compile and link tests.  Line 5 sets up the draco build
environment.  Line 6 tests the \fninety\ compiler and discovers the
module naming convention.  Line 7 finds and verifies the GNU m4
preprocessor.  Line 9 creates the output files.}\label{fig:configure}
\end{figure}

Figure~\ref{fig:configure} shows a sample \texttt{configure.in} using
the \fninety\ macros.  The preamble lines initialize autoconf and give
it a file to check to verify that it is using the proper source
directory, and specify where to find the common scripts needed by
configure.  These files and the aclocal files autoconf uses to create
\texttt{configure} from \texttt{configure.in} are kept in the
\texttt{config/} directory which is a peer of \texttt{src/}.
Figure~\ref{fig:zathras} shows the CVS module entry for Zathras, which
creates the directory \texttt{zathras}, and populates it with the
contents of \texttt{\$CVSROOT/zathras} and
\texttt{\$CVSROOT/draco/config} as \texttt{zathras/config}.  This is
the standard way to import the Draco build system in external project
directories.
\begin{figure}[hbt]
\begin{verbatim}
# Zathras and friends:

Zathras         -a zathras zathras_config
zathras_config  -d zathras/config draco/config
\end{verbatim}
\caption{CVS module entry for Zathras.  The configure files for the
Draco build system located in \texttt{\$CVSROOT/draco/config} are put
into the \texttt{zathras/config} directory on
checkout.}\label{fig:zathras}
\end{figure}

The body of Figure~\ref{fig:configure} contains the macros for
\fninety\ support.  \langfninety\ instructs configure to use \fninety\
for subsequent compiler tests.  \texttt{AC\_DRACO\_ENV} processes
configure arguments, including the selection of \fninety\ compiler and
flags based on either the \withfninety\ argument or the target
platform.  \progfninety\ test the compiler to make sure it can compile
a null program, and discovers how it handles modules.

The last line processes \texttt{Makefile.in} to produce a
\texttt{Makefile} with the appropriate variable substitutions made.

Figure~\ref{fig:makefile} shows a test \texttt{Makefile.in} that
illustrates the use of the \fninety\ variables.  
\begin{figure}[hbt]
\begin{verbatim}
# Macro Definitions

MODSUFFIX       =       @MODSUFFIX@
MODNAME         =       @MODNAME@
MODFLAG         =       @MODFLAG@

F90             =       @F90@
F90FLAGS        =       @F90FLAGS@
F90EXT          =       @F90EXT@
F90FREE         =       @F90FREE@
F90FIXED        =       @F90FIXED@
GM4             =       @GM4@

# Dependency and commands

all:
        @echo "F90       = ${F90}"
        @echo "F90FLAGS  = ${F90FLAGS}"
        @echo "F90EXT    = ${F90EXT}"
        @echo "F90FREE   = ${F90FREE}"
        @echo "F90FIXED  = ${F90FIXED}"
        @echo "MODNAME   = ${MODNAME}"
        @echo "MODSUFFIX = ${MODSUFFIX}"
        @echo "MODFLAG   = ${MODFLAG}"
        @echo "GM4       = ${GM4}"
\end{verbatim}%$
\caption{Sample Makefile.in.  The macro definitions (rhs) are expanded
by the configure script resulting from processing the configure.in
given in Figure~\ref{fig:configure}.  The single target prints the resulting
definitions.}\label{fig:makefile}
\end{figure}

Running autoconf on the \texttt{configure.in} file from
Figure~\ref{fig:configure} and running the resulting
\texttt{configure} file in a directory containing the test
\texttt{Makefile.in} produces a {Makefile} which, when executed,
produces the output in Figure~\ref{fig:make}
\begin{figure}[hbt]
\begin{verbatim}
azathoth $ make
F90       = f90
F90FLAGS  = -O0 -X9 -Am
F90EXT    = f90
F90FREE   = -Free
F90FIXED  = -Fixed
MODNAME   = modname
MODSUFFIX = mod
MODFLAG   = -I
GM4       = m4
\end{verbatim}%$
\caption{Result of sample make on azathoth (a Linux
machine).  The compiler is named f90.  The flags specify level 0
optimaztion, \fninety\ standard, and module generation.  The proper
file extension is .f90.  Free and fixed source are specified with
-Free and -Fixed, respectively.  Module files are named after the
module, with extension .mod.  Module include directories are specified
with -I.  GNU m4 was found, under the name m4.}\label{fig:make}
\end{figure}

\section{Conclusion}

I have written autoconf macros that provide rudimentary \fninety\
support for the Draco build system.  These macros lets a user
configure for \fninety use by finding and verifying a working
\fninety\ compiler, and setting the flags needed for its use as a
makefile command.  Much work remains.  A standard \texttt{Makefile.in}
for \fninety\ source directories, and a tool for ordering dependencies
within and perhaps across directories is needed before Draco truely
supports \fninety\ builds.  

The \langfninety\ and \progfninety\ macros provide templates for
adding other language support in Draco.  In theory, with a minor
modification to \texttt{AC\_LANG\_RESTORE}, a single
\texttt{configure} file could select and configure for several
languages, permitting the use of mixed language directories in Draco.
However, this feature has not been tested; current plans are for all
Draco components (directories) to be unilingual.

Future language support, including improved support for \fninety,
could be made easier by refactoring parts of the existing build
support for \cpp.  This could simplify build macros, make the
build system less \cpp-centric, and improve the chance of having one
template \texttt{Makefile.in} serve all.

\bibliographystyle{plain}
\bibliography{../bib/draco}
\newpage
\closing

\end{document}

%  ========================================================================  %
