%%---------------------------------------------------------------------------%%
%% refactor_VisScope.tex
%% Mike Buksas
%% Time-stamp: <02/07/29 19:03:54 tmonster>
%%---------------------------------------------------------------------------%%
%%
%% Project Vision Statement
%% ------------------------
%% A vision statement explains the project in terms of an end result.
%% It clarifies where the project is going by answering the questions
%%      -What?
%%      -Why?
%%      -How will we know the project is a success?
%%
%% Project Scope Statement
%% -----------------------
%% A scope statement describes how a project will achieve its end
%% result given limited resources.  It sets a project's boundaries by
%% defining the balance between resources, constraints, and
%% deliverables. 
%%
%% Critical Success Factors
%% ------------------------
%% What has to happen for success?  What cannot happen?
%%
%% Risk Assessment
%% ---------------
%% What is the probability of the occurence of each identified risk?
%% What is the impact if it happens?  
%%
%%
%%      ---used with permission and adapted from David A. Schmaltz's 
%%         ``Mastering Projects Workshop: Participant Guide,'' 
%%         True North pgs, Inc., P.O. Box 1532,  Walla Walla, WA 99362, 
%%         projectcommunity.com, Rev. 6, August 2001, copyright.
%%
%%
%% Project Tracking
%% ----------------
%% Track actual results and performance with plans. 
%% 
%%    -Did you change the vision or scope of your project?
%%    -How well did you identify critical success factors?
%%    -What risks did your project realize? ...
%%    -How well were you able to estimate delivery dates?
%%    -What corrective action was taken?
%%    -What lessons have you learned?
%%  
%% The addition of project tracking to the vision and scope statement
%% makes it a living document that is revisited at the completion of
%% the project and whenever risks are realized or vision and scope
%% change. 
%%
%%      ---project tracking section adapted from a LANL ASC Internal 
%%         Assessment of the Jayenne Code Project (CCS-4) by Vicki
%%         Clark and Barbara Hoffbauer (CCN-12).  July 22, 2003. 
%%
%%
%%---------------------------------------------------------------------------%%
\documentclass[11pt]{nmemo}
\usepackage[centertags]{amsmath}
\usepackage{amssymb,amsthm,graphicx}
\usepackage[mathcal]{euscript}
\usepackage{tmadd,tmath}
\usepackage{cite}

%%---------------------------------------------------------------------------%%
%% DEFINE SPECIFIC ENVIRONMENTS HERE
%%---------------------------------------------------------------------------%%
%\newcommand{\elfit}{\ensuremath{\operatorname{Im}(-1/\epsilon(\vq,\omega)}}
%\msection{}-->section commands
%\tradem{}  -->add TM subscript to entry
%\ucatm{}   -->add trademark footnote about entry

%%---------------------------------------------------------------------------%%
%% BEGIN DOCUMENT
%%---------------------------------------------------------------------------%%
\begin{document}

%%---------------------------------------------------------------------------%%
%% OPTIONS FOR NOTE
%%---------------------------------------------------------------------------%%

\toms{Distribution}
%\toms{Joe Sixpak/XTM, MS B226}
\refno{CCS-4:02-????(U)}
\subject{Vision and Scope Statements for Project Camel Norton}

%-------NO CHANGES
\divisionname{Computer and Computational Sciences}
\groupname{CCS-4:Transport Methods Group}
\fromms{Mike Buksas/CCS-4, MS D409}
\phone{(505)66?--????}
\originator{fml}
\typist{fml}
\date{\today}
%-------NO CHANGES

%-------OPTIONS
%\reference{NPB Star Reimbursable Project}
%\thru{P. D. Soran, XTM, MS B226}
%\enc{list}      
%\attachments{list}
%\cy{list}
%\encas
%\attachmentas
%\attachmentsas 
%-------OPTIONS

%%---------------------------------------------------------------------------%%
%% DISTRIBUTION LIST
%%---------------------------------------------------------------------------%%

\distribution {}

%%---------------------------------------------------------------------------%%
%% BEGIN VISION STATEMENT
%%---------------------------------------------------------------------------%%

\opening

\section*{Vision Statement}

Project Camel Norton\footnote{The name is an anagram of Monte Carlo,
  if you must know.} will produce improved code in the Draco MC and
IMC packages. Time permitting, it will also extend to the interaface
components within Milagro. Within IMC, we seek to reduce the amount of
duplicated code in the Particle class heirachy and simplify the
execution of the transport algorithm. This is the highest priority of
this project. In the MC component the goal is to improve the topology
class to reduce the amount of inforation it requires.  This is the
second priority of this project. Within Milagro, we seek to eliminate
the redundancies in the interface code and the global tally. This is
the third priority of this project.

For the remainder of this document, we refer to these as the
particle-transport subproject, the topology subproject and the milagro
interface subproject.

This project is being undertaken to simplify the addition of new
features and improve maintenance. The next area to be affect by the
addition of new features in the transport algorithm. The ease by which
specific features can be added after the refactoring is comlete will
be our criteria for success. This will be done in comparison to the
difficulties encountered in the addition of the surface tracking
capability.

\newpage
\section*{Scope Statement}

%%
%% Include those of the following elements that apply to your project
%%

\subsection*{Product of the Project}

The product of Project Camel Norton will be the replacement of
existing code while preserving functionality. {\em No new capabilities
  will be added as part of this project.}

For the transport algorithm in {\tt draco/imc} the modifications to
the code will likely result in the creation of new classes and the
removal of functionality from the Particle heriachy of classes. We
will use the design in Uncle McFlux as a model, where the Particle
class and the Particle transport algorithm are seperated.

The design for the refactored code will be based on the anticipation
of several new features: DDIMC, Three dimensional support, Material
Motion and Compton Scattering. Our goal is code which will support the
addition of these features with minimum of effort.

The {\tt General\_Topology} class in {\tt draco/mc} supportes a very
general range of cell to domain mappings, including full replication
and fully decomposed meshes. The class, however, does not enforce a
distinction between topologies which contain information about the
global mesh or which only contain the necessary local information
about the neighbors of a given mesh. We seek to improve the topology
class to make this distinction more explicit.

Milagro stuff....

\subsection*{Quality of the Product}

The quality of the new code will be measured informally in terms of
the abstract entities known as {\em code smells}. See~\cite[Chapter
3]{fowler00} for the comprehensive list of smells. An examination of
the code of the Particle classes reveals the following bad smells:
{\em Duplicated Code, Long Method}, and {\em Large Class}.  Based on
the anticipation of new features, we also conclude that the Particle
class suffers from {\em Divergent Change}, that is, it has too many
reasons to change with the addition of new features. 

The changes in the topology class should make the purpose of the
resulting classes easier to understand. The classes should have
greater {\em intention} and be more self-documenting.

Based on the experience of integrating the Surface Tracking
functionality into Wedgehog we also accuse the broader draco library
structure of suffering from {\em Shotgun Surgery}, that is, a new
feature prompted changes in multiple, widely seperated, parts of the
code. This feature was perhaps especially difficult in this regard. We
should be on the lookout for ways to minimize the disruptive impact of
similarly cross-cutting features.

\subsection*{Delivery Date}

\subsection*{Trade-offs}
             E.g., higher product quality implies later delivery date.


\newpage
\section*{Critical Success Factors}

For the particle-transport subproject to be considered a success, we
must be able to recuperate some of the time spent refactoring in time
spent adding new features. Refactoring is strictly a software
development issue, and hence is an ordinary problem. The problem
acquires a dilemma-like character becasue it's goal is diffucilt to
quantify and can only be attempted after the project is complete.

To deal with this dilemma, we can consider incorporating the addition
of a selected feature into a broader project, thereby expanding the
scope to include one cycle of ``refactor, then implement''. The
relative ease of the implementation phase can be used to judge the
success of the project.

Another risk involves keeping the scope of the refactor limited to
areas where the payoff in terms of maintenance is the greatest. The
author of this document has been known to succumb to refactor-mania on
occasion. 

The topology subproject must be evaluated in a similar fashion. It may
be undertaken as the same time as the particle-transport refactor if
it is deemed useful in implementing new features. If it appears to
remain independent, it will be postponed and evaluated seperately.

The Milagro-interface subproject can also be evaluated by its
facilitation of new features. We can also examine it for code smells,
especially code duplication.

\begin{table}[ht]
  \caption{Critical Success Factor Actions.}
    \begin{enumerate}
    \item Resolve it now.
    \item Factor into Plan.
    \item Change Vision/Scope.
    \item Other.
    \end{enumerate}
\end{table}

\begin{table}[ht]
  \begin{center}
    \caption{Critical success factors for Project Camel Norton.}
    \label{tab:critical-success}
    \begin{tabular}{|p{4.5cm}|c|c|p{4.5cm}|} 
    \hline
                       & Problem or & Strategy  &          \\
    Factor             &  Dilemma   & (1,2,3,4) & Comments \\ 
    \hline\hline
    Refactor must pay-off in development time.
    & Dilemma & 3 
    & Consider including a feature implementation step. Keep
    refactoring and implemention stages seperate. \\ 
                                %
    \hline
    \end{tabular}
  \end{center}
\end{table}


\newpage
\section*{Risk Assessment}

\begin{table}[ht]
  \begin{center}
    \caption{Risk Assessment for Project Camel Norton.}
    \label{tab:risk}
    \begin{tabular}{|p{4.5cm}|c|c|c|p{4.5cm}|} 
    \hline
    Risk & Likelihood & Impact & Importance & Contingency \\ 
    \hline\hline
    Too difficult to tell if refactoring is useful & 6 & 3 & 18 &
    Add an implementation step. \\
    \hline
    Refactor spreads to rest of package & 4 & 8 & 32 &
    Evaluate additional refactoring targets on merits \\
    \hline
    Refactored code is significantly slower & 2 & 9 & 18 &
    Add an optimization refactor step when new features are
    implemented. \\ \hline
    \end{tabular}
  \end{center}
\end{table}


\newpage
\section*{Project Tracking}

Track actual results and performance with plans.

   Did you change the vision or scope of your project? 
   
   How well did you identify critical success factors?  What lessons
   have you learned?
 
   What risks did your project realize?  How well did you estimate the
   likelihood and impact of the realized risks?  Did you encounter
   unforseen risks?  What lessons have you learned?
 
   How well were you able to estimate delivery dates?  What lessons
   have you learned?

   What corrective action was taken?
   
   How will you improve the vision and scope statements for future
   projects?

\newpage
\bibliography{cs}
\bibliographystyle{plain}

\closing
\end{document}

%%---------------------------------------------------------------------------%%
%% end of refactor_VisScope.tex
%%---------------------------------------------------------------------------%%

