%%---------------------------------------------------------------------------%%
%% <papername>.tex
%% <user>
%% Time-stamp: <02/07/29 19:03:54 tmonster>
%%---------------------------------------------------------------------------%%
%%
%% Project Vision Statement
%% ------------------------
%% A vision statement explains the project in terms of an end result.
%% It clarifies where the project is going by answering the questions
%%      -What?
%%      -Why?
%%      -How will we know the project is a success?
%%
%% Project Scope Statement
%% -----------------------
%% A scope statement describes how a project will achieve its end
%% result given limited resources.  It sets a project's boundaries by
%% defining the balance between resources, constraints, and
%% deliverables. 
%%
%% Critical Success Factors
%% ------------------------
%% What has to happen for success?  What cannot happen?
%%
%% Risk Assessment
%% ---------------
%% What is the probability of the occurence of each identified risk?
%% What is the impact if it happens?  
%%
%%
%%      ---used with permission and adapted from David A. Schmaltz's 
%%         ``Mastering Projects Workshop: Participant Guide,'' 
%%         True North pgs, Inc., P.O. Box 1532,  Walla Walla, WA 99362, 
%%         projectcommunity.com, Rev. 6, August 2001, copyright.
%%
%%
%% Project Tracking
%% ----------------
%% Track actual results and performance with plans. 
%% 
%%    -Did you change the vision or scope of your project?
%%    -How well did you identify critical success factors?
%%    -What risks did your project realize? ...
%%    -How well were you able to estimate delivery dates?
%%    -What corrective action was taken?
%%    -What lessons have you learned?
%%  
%% The addition of project tracking to the vision and scope statement
%% makes it a living document that is revisited at the completion of
%% the project and whenever risks are realized or vision and scope
%% change. 
%%
%%      ---project tracking section adapted from a LANL ASC Internal 
%%         Assessment of the Jayenne Code Project (CCS-4) by Vicki
%%         Clark and Barbara Hoffbauer (CCN-12).  July 22, 2003. 
%%
%%
%%---------------------------------------------------------------------------%%
\documentclass[11pt]{nmemo}
\usepackage[centertags]{amsmath}
\usepackage{amssymb,amsthm,graphicx}
\usepackage[mathcal]{euscript}
\usepackage{tmadd,tmath}
\usepackage{cite}

%%---------------------------------------------------------------------------%%
%% DEFINE SPECIFIC ENVIRONMENTS HERE
%%---------------------------------------------------------------------------%%
%\newcommand{\elfit}{\ensuremath{\operatorname{Im}(-1/\epsilon(\vq,\omega)}}
%\msection{}-->section commands
%\tradem{}  -->add TM subscript to entry
%\ucatm{}   -->add trademark footnote about entry

%%---------------------------------------------------------------------------%%
%% BEGIN DOCUMENT
%%---------------------------------------------------------------------------%%
\begin{document}

%%---------------------------------------------------------------------------%%
%% OPTIONS FOR NOTE
%%---------------------------------------------------------------------------%%

\toms{Distribution}
%\toms{Joe Sixpak/XTM, MS B226}
\refno{CCS-4:02-????(U)}
\subject{Vision and Scope Statements for the Draco Reorganization Project}

%-------NO CHANGES
\divisionname{Computer and Computational Sciences}
\groupname{CCS-4:Transport Methods Group}
\fromms{<user>/CCS-4, MS D409}
\phone{(505)66?--????}
\originator{fml}
\typist{fml}
\date{\today}
%-------NO CHANGES

%-------OPTIONS
%\reference{NPB Star Reimbursable Project}
%\thru{P. D. Soran, XTM, MS B226}
%\enc{list}      
%\attachments{list}
%\cy{list}
%\encas
%\attachmentas
%\attachmentsas 
%-------OPTIONS

%%---------------------------------------------------------------------------%%
%% DISTRIBUTION LIST
%%---------------------------------------------------------------------------%%

\distribution {}

%%---------------------------------------------------------------------------%%
%% BEGIN VISION STATEMENT
%%---------------------------------------------------------------------------%%

\opening

\section*{Vision Statement}

The Draco Reorganization project will produce a new oragnization of
Draco and Jayenne software components.

We are undertaking this project to ease management and deployment of
Draco and Jayenne code projects. 

Some observable criteria for success:

\begin{itemize}
\item Faster delivery of consistent versions of Jayenne applications on
  multiple platforms.
\item Fewer unnecesary re-deployments of components which have not
  changed (e.g. stable Draco components) 
\item More developer happiness.
\end{itemize}

\newpage
\section*{Scope Statement}

%%
%% Include those of the following elements that apply to your project
%%

\subsection*{Product of the Project}

Various parts of the existing Draco system and Jayeene applications
will be affected by this reorganization. We note some of the
objectives here:

\begin{itemize}
\item Move source control into SourceForge.
\item Promote {\tt draco/mc} and {\tt draco/imc} to Jayenne-level
  projects, or create an appropiate intermediate layer.
\item Create a central repository for bibliographic information that
  is uniformly accessible to documents across various projects.
\item Create a uniform location for automatically generated code
  documentation. Organize the documentation to facilitate linking
  across projects.
\item Create a system for deployment of jayenne builds which use
  consistent draco library components. 
\item Add a profiling system which functions in a similar manner to
  the regression and unit tests. (This will allow for simple
  monitoring of the performace impact of changes in the code)
\end{itemize}

\subsection*{Quality of the Product}

The resulting draco and jayenne projects must retain the ability to
configure and deploy code across multiple platforms. 

\newpage
\section*{Critical Success Factors}

This project will be considered a success when it is in active use by
developers on multiple platforms.

\newpage
\section*{Risk Assessment}

Identify risks.  For each, estimate the likelihood (1-10) of
occurrence and the impact (1-10) if it happens.  The product of these
two values is a measure of the attention the risk deserves.

\begin{table}[ht]
  \begin{center}
    \caption{Risk Assessment for the Draco Reorganization Project.}
    \label{tab:risk}
    \begin{tabular}{|p{4.5cm}|c|c|c|p{4.5cm}|} 
    \hline
    Risk               & Likelihood & Impact & Importance & Contingency\\ 
    \hline\hline
    Reorg fails to deliver adequate benefit for time invested. 
    & 2 & 4 & 8 & Scale back on optional activities. Get back to
    work. \\
    \hline
    \end{tabular}
  \end{center}
\end{table}


\newpage
\section*{Project Tracking}

Track actual results and performance with plans.

   Did you change the vision or scope of your project? 
   
   How well did you identify critical success factors?  What lessons
   have you learned?
 
   What risks did your project realize?  How well did you estimate the
   likelihood and impact of the realized risks?  Did you encounter
   unforseen risks?  What lessons have you learned?
 
   How well were you able to estimate delivery dates?  What lessons
   have you learned?

   What corrective action was taken?
   
   How will you improve the vision and scope statements for future
   projects?

\closing
\end{document}

%%---------------------------------------------------------------------------%%
%% end of <papername>.tex
%%---------------------------------------------------------------------------%%

