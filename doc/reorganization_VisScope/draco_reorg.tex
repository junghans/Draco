%%---------------------------------------------------------------------------%%
%% draco_reorg.tex
%% Mike Buksas
%% Time-stamp: <02/07/29 19:03:54 tmonster>
%%---------------------------------------------------------------------------%%
%%
%% Project Vision Statement
%% ------------------------
%% A vision statement explains the project in terms of an end result.
%% It clarifies where the project is going by answering the questions
%%      -What?
%%      -Why?
%%      -How will we know the project is a success?
%%
%% Project Scope Statement
%% -----------------------
%% A scope statement describes how a project will achieve its end
%% result given limited resources.  It sets a project's boundaries by
%% defining the balance between resources, constraints, and
%% deliverables. 
%%
%% Critical Success Factors
%% ------------------------
%% What has to happen for success?  What cannot happen?
%%
%% Risk Assessment
%% ---------------
%% What is the probability of the occurence of each identified risk?
%% What is the impact if it happens?  
%%
%%
%%      ---used with permission and adapted from David A. Schmaltz's 
%%         ``Mastering Projects Workshop: Participant Guide,'' 
%%         True North pgs, Inc., P.O. Box 1532,  Walla Walla, WA 99362, 
%%         projectcommunity.com, Rev. 6, August 2001, copyright.
%%
%%
%% Project Tracking
%% ----------------
%% Track actual results and performance with plans. 
%% 
%%    -Did you change the vision or scope of your project?
%%    -How well did you identify critical success factors?
%%    -What risks did your project realize? ...
%%    -How well were you able to estimate delivery dates?
%%    -What corrective action was taken?
%%    -What lessons have you learned?
%%  
%% The addition of project tracking to the vision and scope statement
%% makes it a living document that is revisited at the completion of
%% the project and whenever risks are realized or vision and scope
%% change. 
%%
%%      ---project tracking section adapted from a LANL ASC Internal 
%%         Assessment of the Jayenne Code Project (CCS-4) by Vicki
%%         Clark and Barbara Hoffbauer (CCN-12).  July 22, 2003. 
%%
%%
%%---------------------------------------------------------------------------%%
\documentclass[11pt]{nmemo}
\usepackage[centertags]{amsmath}
\usepackage{amssymb,amsthm,graphicx}
\usepackage[mathcal]{euscript}
\usepackage{tmadd,tmath}
\usepackage{cite}
\usepackage{c++}

%%---------------------------------------------------------------------------%%
%% DEFINE SPECIFIC ENVIRONMENTS HERE
%%---------------------------------------------------------------------------%%
%\newcommand{\elfit}{\ensuremath{\operatorname{Im}(-1/\epsilon(\vq,\omega)}}
%\msection{}-->section commands
%\tradem{}  -->add TM subscript to entry
%\ucatm{}   -->add trademark footnote about entry

%%---------------------------------------------------------------------------%%
%% BEGIN DOCUMENT
%%---------------------------------------------------------------------------%%
\begin{document}

%%---------------------------------------------------------------------------%%
%% OPTIONS FOR NOTE
%%---------------------------------------------------------------------------%%

\toms{Distribution}
\refno{CCS-4:04-????(U)}
\subject{Vision and Scope Statements for Project Thuban}

%-------NO CHANGES
\divisionname{Computer and Computational Sciences}
\groupname{CCS-4:Transport Methods Group}
\fromms{Mike Buksas/CCS-4, MS D409}
\phone{(505)667--7580}
\originator{fml}
\typist{fml}
\date{\today}
%-------NO CHANGES

%%---------------------------------------------------------------------------%%
%% DISTRIBUTION LIST
%%---------------------------------------------------------------------------%%

\distribution {
  M.W. Buksas, CCS-4, D-409 \\
  J.D. Densmore, CCS-4, D-409 \\
  T.M. Evans, CCS-4, D-409 \\
  T.J. Urbatsch, CCS-4, D-409 \\
  CCS-4 Files.
}

%%---------------------------------------------------------------------------%%
%% BEGIN VISION STATEMENT
%%---------------------------------------------------------------------------%%

\opening

\section*{Vision Statement}

Project Thuban \footnote{Thuban, also known as {\em Alpha Draconis},
  is a fourth-magnitude star in the constellation Draco. It is notable
  for being the star closest to the north pole from approximately 3000
  BC to 1900 BC, due to the procession of the Earth's axis.}  will add
new components to Draco, transforming it into a central repository for
the common \C++ software components, the build system, the
documentation system, (including autodoc and LaTeX document
templates), LaTeX extensions, papers, presentations, and bibliography
files.

We are undertaking this project to improve access to all of the
affected components by improving and unifying their organization. For
example, there is little sharing of bibliography information because
bibliography files tend to be housed with documents that use them in
the scattered {\tt archive} respositoy. This has led to duplication of
bibliographic entries.

Some observable criteria for success:

\begin{itemize}
\item Increased re-use of bibliographic information with less
  duplication.
\item Increased re-use of contrubuted LaTeX styles and classes.
\item Greater group-wide involvement in contributions to the common
  document repository, LaTeX extensions and envrionment improvements.
\item More developer and customer happiness. Birds and cute bunny
  rabbits appear and spontaneously break into song.
\end{itemize}

\section*{Scope Statement}

%%
%% Include those of the following elements that apply to your project
%%

\subsection*{Product of the Project}

The product of project Thuban is an enlarged and more useful Draco
respository. Draco should be more useful for developers who actively
participate in it's evolution and users, who are simply interested in
it's services. To this end, we propose the following products:

\begin{itemize}
  \item A central, organized, document respoitiory within Draco in
    which it is easy to find documents and easy to determine where new
    documents should be added.
  \item A central, organized, bibliography repository within Draco in
    which it is easy to find references and easy to determine where
    new references should be added.
  \item Externally visible images of the parts of Draco repository
    which are of interest to users. E.g. The elisp components for
    xemacs, the document templates, the bibliographies, etc.. should
    have stable locations under {\tt /codes/radtran}. Users can be
    directed to set environment variables (e.g. {\tt \$TEXINPUTS,
      \$BIBINPUTS}) and other configurations (e.g. {\tt
      .xemacs/init.el}) to these stable values.
\end{itemize}

\section*{Critical Success Factors}

The goal of this project is improved organization, and from that,
increased productivity. Either of these are diffucult to measure
quantatively, although we can rely on a shared notion of
``organizational esthetics'' to determine whreter the new Draco is
better organized and more useful than the old Draco + {\tt archive}.

So, in order to succeed, this project must be based on a common vision
of what constitutes better organization. Attaining this shared vision,
will require reaching a concensus and adjusting the scope of this
project accordingly. 

\begin{table}[ht]
  \begin{center}
    \caption{Critical success factors for Project Thuban.}
    \label{tab:critical-success}
    \begin{tabular}{|p{4.5cm}|c|c|p{4.5cm}|} 
    \hline
    Factor             &  Character   & Strategy & Comments \\ 
    \hline\hline
    Must attain shared vision of ``organization'' & 
    Dilemma & Change Vision/Scope &
    Discuss alternatives, ask users, compromise and remember that we
    can always change it later. \\ \hline
    \end{tabular}
  \end{center}
\end{table}

\section*{Risk Assessment}

\begin{table}[ht]
  \begin{center}
    \caption{Risk Assessment for the Project Thuban.}
    \label{tab:risk}
    \begin{tabular}{|p{4.5cm}|c|c|c|p{4.5cm}|} 
    \hline
    Risk & Likelihood & Impact & Importance & Contingency \\
    \hline\hline
    Reorg fails to deliver adequate benefit for time invested. &
    2 & 4 & 8 & 
    Scale back on optional activities. Get back to  work. \\ \hline
                                %
    Draco repository becomes too large to find things
    easily. Confusion over it's multiple parts arises. & 
    5 & 5 & 25 &
    Consider splitting into sub-repositories or completely seperate
    repositories. \\ \hline
    \end{tabular}
  \end{center}
\end{table}

\section*{Project Planning}

Because this project is small in scope, we include the planning
details here, rather then in a seperate document.

\subsubsection*{Plan highlights}

\begin{itemize}
\item Do not begin until Project CLUBIMC is completed and the current
  Draco repositiy is stable.

\item Create a documents and bibliography repository in Draco.
  
\item Move items from the document directories in {\tt archive} into
  the documents and bibliography repository in Draco. Fix dcocuments
  broken by changing paths.

\item Re-locate the contributed LaTeX classes from {\tt archive} to
  Draco.
  
\item Create externally visible copies of the repositories useful to
  users. This way, a user has access to non-developer parts of Draco
  without checking out the archive.

  This can be done with a CVS checkout, or by creating a copy of the
  directory itself.

\end{itemize}

\subsubsection*{Proposed directory structure}

{\ttfamily
\begin{tabbing}
  \hspace*{0.5cm}\=\hspace{0.5cm}\=\hspace{0.5cm}\=\hspace{0.5cm}\=\kill
  draco/ \\
  \> src/ ({\rmfamily as before, minus {\tt mc} and {\tt imc}}) \\
  \> \> ds++/ \\
  \> \> \> doc/ {(\rmfamily package-specific documentation still with
    package, e.g. release notes)} \\
  \> \> c4/ \\
  \> \> {\rmfamily etc...} \\
  \> documents/ ({\rmfamily contents from {\tt mcdoc} and {\tt imcdoc}.)}\\
  \> \> bibfiles/ \\
  \> \> figures/ ({\rmfamily central location for widely used figures only)}\\
  \> \> research\_notes/ \\
  \> \> papers/ \\
  \> \> presentations/ \\
  \> \> planning/ \\
  \> \> methods/ \\
  \> environment/ \\
  \> \> templates/ \\
  \> \> elisp/ \\
  \> \> latex/ \\
\end{tabbing}
}    

\subsubsection*{Notes}

\begin{itemize}

\item The proposed organization within the {\tt documents\ } directory
  is by type, rather than subject matter. My impression is that this
  will make things easier to find, but this is debateable.

\item The {\tt bibfiles} directory should contain the {\em
    autobibliography} of everything in the other documents
  directories. Citations of outside work can be grouped into a small
  number of well-defined topics, e.g. {\em imc}, {\em mc}, {\em
    comp-sci}, etc...

\end{itemize}
  


\section*{Project Tracking}

%Track actual results and performance with plans.
%
%   Did you change the vision or scope of your project? 
%   
%   How well did you identify critical success factors?  What lessons
%   have you learned?
% 
%   What risks did your project realize?  How well did you estimate the
%   likelihood and impact of the realized risks?  Did you encounter
%   unforseen risks?  What lessons have you learned?
% 
%   How well were you able to estimate delivery dates?  What lessons
%   have you learned?
%
%   What corrective action was taken?
%   
%   How will you improve the vision and scope statements for future
%   projects?

\subsection*{Change log:}

\begin{description}
\item[3/25/04] Submitted initial version for review.

\end{description}

\closing
\end{document}

%%---------------------------------------------------------------------------%%
%% end of draco_reorg.tex
%%---------------------------------------------------------------------------%%

