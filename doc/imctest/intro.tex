%%---------------------------------------------------------------------------%%
%% introduction for imc-dev manual
%%---------------------------------------------------------------------------%%

\section{Introduction}

The \imctest\ package represents a first step towards the development
of \milagro.  In this sense, \imctest\ may be considered an early
release of \milagro.  The goals of \imctest\ are:
\begin{enumerate}
\item develop an extensible \cpp\ object-oriented/generic design;
\item test the implementation of the IMC algorithm on several simple
  test problems;
\item test the parallel performance of IMC on several platforms.
\end{enumerate}
The object-oriented/generic design of \imctest\ will ensure an easy
transition to the full \milagro\ package.  The improvements and
additions that are required to promote \imctest\ to \milagro\ include:
\begin{enumerate}
\item interfaces to various host-codes;
\item improved physics;
\item new mesh-types and geometries;
\end{enumerate}
The classes and data structures in \imctest\ are easily extensible;
thus, the classes used in \imctest\ should work in \milagro. For more
information, the \jayenne\ code development plan is described in
detail in Ref.~\citen{xtm:rn98xxx}.

The purpose of this manual is to document the code structure of the
\imctest\ package for users of the \imctest\ library.  In essence, this
manual contains the following information:
\begin{itemize}
\item a listing and description of the classes and data structures
  which comprise the \imctest\ library;
\item descriptions of the interactions between classes and types in
  \imctest;
\item descriptions of the interfaces for each class in \imctest.
\end{itemize}
The information presented herein will facilitate easy usage of the
\imctest\ library by code developers in X-division.  It will also serve
as a reference for users of the code.

