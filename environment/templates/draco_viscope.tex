%%---------------------------------------------------------------------------%%
%% <papername>.tex
%% <user>
%% Time-stamp: <02/07/29 19:03:54 tmonster>
%% Copyright � 2006 Los Alamos National Security, LLC
%%---------------------------------------------------------------------------%%
%%
%% Project Vision Statement
%% ------------------------
%% A vision statement explains the project in terms of an end result.
%% It clarifies where the project is going by answering the questions
%%      -What?
%%      -Why?
%%      -How will we know the project is a success?
%%
%% Project Scope Statement
%% -----------------------
%% A scope statement describes how a project will achieve its end
%% result given limited resources.  It sets a project's boundaries by
%% defining the balance between resources, constraints, and
%% deliverables. 
%%
%% Critical Success Factors
%% ------------------------
%% What has to happen for success?  What cannot happen?
%%
%% Risk Assessment
%% ---------------
%% What is the probability of the occurrence of each identified risk?
%% What is the impact if it happens?  
%%
%%
%%      ---used with permission and adapted from David A. Schmaltz's 
%%         ``Mastering Projects Workshop: Participant Guide,'' 
%%         True North pgs, Inc., P.O. Box 1532,  Walla Walla, WA 99362, 
%%         projectcommunity.com, Rev. 6, August 2001, copyright.
%%
%%
%% Project Tracking
%% ----------------
%% Track actual results and performance with plans. 
%% 
%%    -Did you change the vision or scope of your project?
%%    -How well did you identify critical success factors?
%%    -What risks did your project realize? ...
%%    -How well were you able to estimate delivery dates?
%%    -What corrective action was taken?
%%    -What lessons have you learned?
%%  
%% The addition of project tracking to the vision and scope statement
%% makes it a living document that is revisited at the completion of
%% the project and whenever risks are realized or vision and scope
%% change. 
%%
%%      ---project tracking section adapted from a LANL ASC Internal 
%%         Assessment of the Jayenne Code Project (CCS-4) by Vicki
%%         Clark and Barbara Hoffbauer (CCN-12).  July 22, 2003. 
%%
%%
%%---------------------------------------------------------------------------%%
\documentclass[11pt]{nmemo}
\usepackage[centertags]{amsmath}
\usepackage{amssymb,amsthm,graphicx}
\usepackage[mathcal]{euscript}
\usepackage{tmadd,tmath}
\usepackage{cite}
\usepackage{threeparttable}

%%---------------------------------------------------------------------------%%
%% DEFINE SPECIFIC ENVIRONMENTS HERE
%%---------------------------------------------------------------------------%%
%\newcommand{\elfit}{\ensuremath{\operatorname{Im}(-1/\epsilon(\vq,\omega)}}
%\msection{}-->section commands
%\tradem{}  -->add TM subscript to entry
%\ucatm{}   -->add trademark footnote about entry

%%---------------------------------------------------------------------------%%
%% BEGIN DOCUMENT
%%---------------------------------------------------------------------------%%
\begin{document}

%%---------------------------------------------------------------------------%%
%% OPTIONS FOR NOTE
%%---------------------------------------------------------------------------%%

\toms{Distribution}
%\toms{Joe Sixpak/XTM, MS B226}
\refno{CCS-4:02-????(U)}
\subject{Vision and Scope Statements for the <papername> Project}

%-------NO CHANGES
\divisionname{Computer and Computational Sciences}
\groupname{CCS-2:Computational Physics \& Methods}
\fromms{<user>/CCS-2, MS D409}
\phone{(505)66?--????}
\originator{fml}
\typist{fml}
\date{\today}
%-------NO CHANGES

%-------OPTIONS
%\reference{NPB Star Reimbursable Project}
%\thru{P. D. Soran, XTM, MS B226}
%\enc{list}      
%\attachments{list}
%\cy{list}
%\encas
%\attachmentas
%\attachmentsas 
%-------OPTIONS

%%---------------------------------------------------------------------------%%
%% DISTRIBUTION LIST
%%---------------------------------------------------------------------------%%

\distribution {}

%%---------------------------------------------------------------------------%%
%% BEGIN VISION STATEMENT
%%---------------------------------------------------------------------------%%

\opening

\section*{Vision Statement}

-What will the project produce?

-Why are we doing this project?

-What are the observable and measurable criteria for success?


\newpage
\section*{Scope Statement}

%%
%% Include those of the following elements that apply to your project
%%

\subsection*{Product of the Project}

\subsection*{Quality of the Product}

\subsection*{Delivery Date}

\subsection*{Quantity of Product}

\subsection*{Required Resources}
             E.g., FTEs, hardware, software, anything from outside the
             team and not controlled by the team,...

\subsection*{Trade-offs}
             E.g., higher product quality implies later delivery date.

\subsection*{Graphical Representation}
             E.g.,   
             \begin{verbatim}

                           Features
                              /\
                             /  \
                            /    \
                        Cost ---- Schedule

              \end{verbatim}


\newpage
\section*{Critical Success Factors}

What has to happen for your project to be a success?  What cannot
happen?

%What is the source of your critical success factor?
%\begin{itemize}
%  \item Business
%  \item People
%  \item Technology
%  \item Politics
%\end{itemize}

%Is the critical success factor a problem or dilemma?
%\begin{itemize}
%  \item A problem has a definite solution...solve it and move on.
%  \item A dilemma doesn't have a unique solution...deal with it now
%        and revisit it later.
%\end{itemize}

%How will you deal with the critical success factor?
%\begin{enumerate}
%  \item Resolve it now.
%  \item Factor into Plan.
%  \item Change Vision/Scope.
%  \item Other.
%\end{enumerate}

\begin{center}
  \begin{threeparttable}
    \caption{Critical success factors for the <papername> Project.}
    \label{tab:critical-success}
    \begin{tabular}{|p{4.5cm}|c|c|c|p{4.5cm}|} 
      \hline
      & Problem or        &  &  \\
      Factor             &  Dilemma\tnote{a} & Source\tnote{b} & Strategy\tnote{c} & Comments \\ 
      \hline\hline
      New method is bad. &  Dilemma   &    2      
      & probably doesn't resolve bnd layers \\
      \hline
    \end{tabular}
    \begin{tablenotes}
    \item[a] \footnotesize{Is the critical success factor a problem or a
        dilemma.  A problem has a definite solution.  Solve it and move
        on.  A dilemma doesn't have a unique solution.  deal with it now
        and revisit it later.}
    \item[b] The source of a critical success factor is one of: Business,
      People, Technology, Politics.
    \item[c] \footnotesize{How will you deal with the critical success
        factor? (1) Resolve it now, (2) Factor into Plan, (3) Change
        Vision/Scope, or (4) Other.}
    \end{tablenotes}
  \end{threeparttable}
\end{center}
    
\newpage
\section*{Risk Assessment}

Identify risks.  For each, estimate the likelihood (1-10) of
occurrence and the impact (1-10) if it happens.  The product of these
two values is a measure of the attention the risk deserves.

\begin{table}[ht]
  \begin{center}
    \caption{Risk Assessment for the <papername> Project.}
    \label{tab:risk}
    \begin{tabular}{|p{4.5cm}|c|c|c|p{4.5cm}|} 
    \hline
    Risk & Likelihood & Impact & Importance & Contingency \\
    \hline\hline
    Method doesn't resolve bnd layers &
    8 & 9 & 72 & 
    if it's faster, combine with old method; if not, abandon \\
    \hline
    \end{tabular}
  \end{center}
\end{table}


\newpage
\section*{Project Tracking}

Track actual results and performance with plans.

   Did you change the vision or scope of your project? 
   
   How well did you identify critical success factors?  What lessons
   have you learned?
 
   What risks did your project realize?  How well did you estimate the
   likelihood and impact of the realized risks?  Did you encounter
   unforeseen risks?  What lessons have you learned?
 
   How well were you able to estimate delivery dates?  What lessons
   have you learned?

   What corrective action was taken?
   
   How will you improve the vision and scope statements for future
   projects?

\closing
\end{document}

%%---------------------------------------------------------------------------%%
%% end of <papername>.tex
%%---------------------------------------------------------------------------%%

