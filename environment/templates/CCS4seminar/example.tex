\documentclass[SRD,
%               article,
               slidesonly,
                notes
%               notesonly
              ]{CCS4seminar}

\title{Example slides for CCS4seminar class}
\author{Michael Buksas}
\date{\today}

\slideplacement{here*}
\SRDtext{\textcolor{red}{XXXXXX/XX}}

\begin{document}

\begin{slide} \seminarTitle \end{slide}

\begin{slide}

  \slideheading{CCS4seminar class}
  
  The CCS4seminar class is built on the {\tt seminar} class by Timothy
  van Zandt and uses the {\tt fancyhdr} style.
  
  It adds support for labeling of classified documents, and ``M of N''
  style page numbers.

  It also creates a custom title page with the comand:
  \begin{verbatim}
      \begin{slide} \seminarTitle \end{slide}
  \end{verbatim}
  The LANL and CCS4 logos are automatically included. To change this
  behavior, redefine the \verb=\seminarTitle= command, or just roll
  your own title page

\end{slide}

\begin{slide}

  \slideheading{The Seminar Class}

  The seminar class is a replacement for the \LaTeX\ slide class. It
  offers more customization of the appearance, section and subsection
  headings, note generation and printing, and a lot more.
  
  \slidesubheading{Notes} \label{slide:notes}
  
  Note and slide generation is controlled through class arguments. Use
  {\tt slidesonly} and {\tt notesonly} to produce them seperately.
  {\tt article} Produces two slides per page {\em or} one slide and
  notes.

\end{slide}

These are notes for Slide~\pageref{slide:notes}. They are part of the
same \LaTeX\ document as the slides and are located outside of the
slide environment.

\begin{slide}
  
  The {\tt seminar} class is based on the \LaTeX\ {\tt article} class.
  Everything works pretty much as you would expect, like equations:

\begin{eqnarray}
  \frac{\partial B}{\partial t} & = & \nabla\times E \nonumber \\
  \frac{\partial D}{\partial t} & = & -\nabla\times H
  \label{eqn:maxwell} \\
  \nabla\cdot D = \rho & & \nabla\cdot B = 0 \nonumber
\end{eqnarray}

The {\tt verbatim} environment. (Good for code snippets)
\begin{verbatim}
  pair<double,int> distance_bin = 
      Angular_Distance_Computer<CS>::compute_distance(
          particle, angular_mesh);
  double angular_bin_distance = distance_bin.first;
\end{verbatim}

\end{slide}

\begin{slide}

\slidesubheading{Figures and references}

And the figure environment:

\begin{figure}
  \begin{center}
    \scalebox{0.7}{\input{figures/multiple_paths.pstex_t}}
  \end{center}
  \caption{A random figure}
  \label{fig:figure}
\end{figure}

\end{slide}

\begin{slide}

  \slidesubheading{Figure sizes}

  The {\tt seminar} class uses {\em magnification} to produce
  slide-sized text and to keep proportions the same between the slides
  and the article format. This also affects image sizes.
  
  Use \verb=\semin=\ and \verb=\semcm=\ for slide-sized inches and cm
  to specify image sizes.  \verb=\scalebox= can be used to enclose and
  scale images as well.

  \slidesubheading{References}

  The \verb=\pageref=\ command works with slide numbers:
  ``\verb=Figure~\ref{fig:figure} on Slide~\pageref{fig:figure}=''
  yields ``Figure~\ref{fig:figure} on Slide~\pageref{fig:figure}''.

  (This is perhaps of dubious utility in a slide presentation.)
  
\end{slide}

\begin{slide}

  \slidesubheading{Orientation}
  
  Class {\tt seminar} produces slides in Landscape orientation by
  default. To make slides in portrait orientation, use the {\tt
    portrait} class argument and the {\tt slide*} environment. e.g.
\begin{verbatim}
 \documentclass[slidesonly,portrait]{CCS4seminar}
 \begin{document}
 \begin{slide*}
   Blah, blah, blah...
 \end{slide*}
\end{verbatim}

\end{slide}      

\begin{slide}

  \slidesubheading{Slide Borders}

  We can also change the border of the slide with the
  \verb=\slideframe{}= command. It's argument values are: {\tt plain,
    oval, shadow, double, none}. Type {\tt plain} is the default (see
  previous pages). 

  A \verb=\slideframe= command in the preamble changes the default.

  The \verb=\slideframe=\ command also has an optional argument for 
  commands which affect the style.

  This page was created with \verb=\slideframe{oval}=.
  \slideframe{oval}
  
\end{slide}

\begin{slide}

  \slideframe{shadow}
  This slide uses \verb=\slideframe{shadow}=

  \slideheading{Slide headings}

  Slide headings and sub-headings are defined with
  \verb=\slideheading= and \verb=\slidesubheading=.

  They appear identically on the slide. \verb=\slideheading=\ updates
  the section number and changes the footnote entry.

\end{slide}

\begin{slide}

  \slideframe{double}

  This slide uses \verb=\slideframe{double}=.

  \slideheading{Headers/Footers}

  \renewcommand{\slideheadfont}{\bfseries}
  \renewcommand{\slidefootfont}{\bfseries}

  \leftFoot {Left Footer }
  \rightFoot{Right Footer}
  \leftHead {Left Header }
  \rightHead{Right Header}

  There are four commands for changing the slide headings.

  \begin{center}
    \begin{tabular}{|l|l|} \hline
      Command & Default Value \\ \hline 
      \verb=\leftHead{}=  & \verb=\theAuthor= \\
      \verb=\rightHead{}= & \verb=\theDate=   \\
      \verb=\leftFoot{}=  & \verb=\theslidesection--\theslideheading= \\
      \verb=\rightFoot{}= & \verb=\theSlideOfTotal=  \\ \hline
    \end{tabular}
  \end{center}
  
  Changes are local to the slide they are defined in. If outside of a
  slide environment, they affect all subsequent slides.

\end{slide}

\begin{slide}

  The slide sections which appear in the footer are defined with the
  \verb=\slideheading= command. The section numbers are assigned
  automatically, starting with one.

  The command \verb=\theSlideOfTotal= generates the ``\theSlideOfTotal''\
  page numbers. To define your own, the slide number is available as
  \verb=\theslide= and the final slide number is \verb=\lastSlide=.

\end{slide}

\begin{slide}
  \slideheading{SRD tags}

  This presentation was created with the optional SRD argument. This
  causes all slides to be labeled as secret.

  \begin{verbatim}
    \documentclass[slidesonly,SRD]{CCS4seminar}
  \end{verbatim}
  The text of the SRD label is changed with the command
  \verb=\SRDtext{}=. 
  
  We have done this to avoid splashing ``SECRET/RD'' around in an
  unclassified presentation

\end{slide}

\begin{slide} \notSRD
  This slide contains the command \verb=\notSRD= and so is labeled
  UNCLASSIFIED. Only this slide is affected, since the command was
  issued inside the slide environment.
  
  The text label for unclassified pages can be changed with \\
  \verb=\notSRDtext{}=.
  
  If \verb=\notSRD= is issued outside a slide environment, it affects
  all remaining slides, except when overridden by \verb=\isSRD=.

\end{slide}

\begin{slide}

  \slideheading{Documentation}

  For additional information on the options available in the seminar
  class and the fancyhdr style on which CCS4seminar is based, see the
  files: 

  {\tt /codes/radtran/vendors/tex/doc/sem-user.pdf} and
  {\tt /codes/radtran/vendors/tex/doc/fancyhdr-1.pdf}.

  The files to create this document are stored in

  {\tt /codes/radtran/vendors/tex/doc/CCS4seminar}.

 \end{slide}
   

\end{document}

