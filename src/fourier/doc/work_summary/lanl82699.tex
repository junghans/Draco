%%---------------------------------------------------------------------------%%
%% lanl82699.tex
%% John Gulick
%% $Id$
%%---------------------------------------------------------------------------%%
\documentclass[11pt]{rnote}
\usepackage[centertags]{amsmath}
\usepackage{amssymb,amsthm,graphicx}
\usepackage[mathcal]{euscript}
\usepackage{tabularx}
\usepackage{cite}
%\usepackage{c++}
%\usepackage{tmadd,tmath}
%\usepackage{stl}

%%---------------------------------------------------------------------------%%
%% DEFINE SPECIFIC ENVIRONMENTS HERE
%%---------------------------------------------------------------------------%%
%\newcommand{\elfit}{\ensuremath{\operatorname{Im}(-1/\epsilon(\vq,\omega)}}
%\msection{}-->section commands
%\tradem{}  -->add TM subscript to entry
%\ucatm{}   -->add trademark footnote about entry

%%---------------------------------------------------------------------------%%
%% BEGIN DOCUMENT
%%---------------------------------------------------------------------------%%
\begin{document}

%%---------------------------------------------------------------------------%%
%% OPTIONS FOR NOTE
%%---------------------------------------------------------------------------%%

\toms{Todd Urbatsch, Tom Evans, Todd Wareing, Jim Morel, and Gordon
  Olson, XTM, D409}
\refno{XTM:99-60 (U)}
\subject{Summary of Completed Work for John C. Gulick}

%-------NO CHANGES
\divisionname{Applied Theoretical \& Computational Physics Div.}
\groupname{X-TM:Transport Methods Group}
\fromms{John Gulick/XTM D409}
\phone{(505)667--5790}
\originator{gulick}
\typist{gulick}
\date{August 27, 1999}
%-------NO CHANGES

%-------OPTIONS
%\reference{NPB Star Reimbursable Project}
%\thru{P. D. Soran, XTM, MS B226}
%\enc{list}      
%\attachments{list}
%\cy{list}
%\encas
%\attachmentas
%\attachmentsas 
%-------OPTIONS

%%---------------------------------------------------------------------------%%
%% DISTRIBUTION LIST
%%---------------------------------------------------------------------------%%

\distribution {}

%%---------------------------------------------------------------------------%%
%% BEGIN NOTE
%%---------------------------------------------------------------------------%%

\opening

\noindent
The following is a summary of the work completed for Summer FY 99 by
John C. Gulick, X-TM.
\section{Introduction}
\noindent
During the summer of 1999 a Fourier analysis package was designed,
implemented and tested.  The purpose of creating this package is to assist in
the developement and analysis of new transport discretizations and
acceleration techniques of interest to X-TM.  By providing an easily
extensible framework to Fourier analyze new schemes the capability of
quickly implementing new techniques becomes possible.  For the development of this
package there were three important phases:
\begin{enumerate}
 \item Design
 \item Development and Implementation
 \item Testing
\end{enumerate}
\section{Design}
\noindent
The design of the project involved the creation of a requirements
document which served to guide the project.  The requirements document 
was a key piece of material as it detailed all of the following three
requirement areas:
\begin{description}
 \item [Technical Requirements] Requirements having to do with the
   Fourier analysis capability, specifications and scope of the project.
 \item [Software Requirements] Requirements having to do with the
   programming/coding aspects of the Fourier analysis package.
 \item [Testing Requirements] Requirements having to do with the
 testing and verification of the Fourier analysis package.
\end{description}
The creation of this document detailed the requirements of the project 
and the exact steps that the project would go through in fulfilling
those requirements.
\section{Development and Implementation}
\noindent
For the development and implementation it was decided to try and use 
as much ``off the shelf'' software as possible.  This idea of
``software reuse'' allowed the actual coding of the package to be
staight forward.  In the current form there are three main software
components that are being reused.
\begin{description}
 \item [MTL] The Matrix Template Library is a set of very powerful
   direct and iterative linear algebra operations.
 \item [LAPACK] A library of routines that interfaces with the MTL to
   provide some very important linear algebra functionality.
 \item [PGPLOT] A two and three dimensional plotting library for
   creating graphics during code execution.
\end{description} 
The package was written in C++ and many of the modern capabilities of the
language were used.  Examples of these capabilities were:
\begin{enumerate}
 \item Object-Oriented Programming Techniques
 \item Generic Programming (STL/MTL) Techniques
 \item Templating Techniques
\end{enumerate}
\section{Testing}
The package was tested with discretizations and acceleration
techniques with known results.  By comparing the results of the code
with published data the proper operation of the package could be
verified.  Several MAPLE and FORTRAN routines were built to further
verify the new package.
Currently the package has been tested for Simple Corner Balance/Lumped 
Linear Discontinuous with and without ``Modified 4-Step'' DSA,
Upstream Corner Balance with and without ``Modified 4-Step'' DSA, and
Linear Characteristics with no acceleration.
\section{Future Work}
Although first version of the Fourier package has been completed there 
is a great deal of 
\closing
\end{document}

%%---------------------------------------------------------------------------%%
%% end of lanl82699.tex
%%---------------------------------------------------------------------------%%
