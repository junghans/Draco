%%---------------------------------------------------------------------------%%
%% particle_transport_refactor.tex
%% Mike Buksas
%% Time-stamp: <02/07/29 19:03:54 tmonster>
%%---------------------------------------------------------------------------%%
%%
%% Project Vision Statement
%% ------------------------
%% A vision statement explains the project in terms of an end result.
%% It clarifies where the project is going by answering the questions
%%      -What?
%%      -Why?
%%      -How will we know the project is a success?
%%
%% Project Scope Statement
%% -----------------------
%% A scope statement describes how a project will achieve its end
%% result given limited resources.  It sets a project's boundaries by
%% defining the balance between resources, constraints, and
%% deliverables. 
%%
%% Critical Success Factors
%% ------------------------
%% What has to happen for success?  What cannot happen?
%%
%% Risk Assessment
%% ---------------
%% What is the probability of the occurence of each identified risk?
%% What is the impact if it happens?  
%%
%%
%%      ---used with permission and adapted from David A. Schmaltz's 
%%         ``Mastering Projects Workshop: Participant Guide,'' 
%%         True North pgs, Inc., P.O. Box 1532,  Walla Walla, WA 99362, 
%%         projectcommunity.com, Rev. 6, August 2001, copyright.
%%
%%
%% Project Tracking
%% ----------------
%% Track actual results and performance with plans. 
%% 
%%    -Did you change the vision or scope of your project?
%%    -How well did you identify critical success factors?
%%    -What risks did your project realize? ...
%%    -How well were you able to estimate delivery dates?
%%    -What corrective action was taken?
%%    -What lessons have you learned?
%%  
%% The addition of project tracking to the vision and scope statement
%% makes it a living document that is revisited at the completion of
%% the project and whenever risks are realized or vision and scope
%% change. 
%%
%%      ---project tracking section adapted from a LANL ASC Internal 
%%         Assessment of the Jayenne Code Project (CCS-4) by Vicki
%%         Clark and Barbara Hoffbauer (CCN-12).  July 22, 2003. 
%%
%%
%%---------------------------------------------------------------------------%%
\documentclass[11pt]{nmemo}
\usepackage[centertags]{amsmath}
\usepackage{amssymb,amsthm,graphicx}
\usepackage[mathcal]{euscript}
\usepackage{tmadd,tmath}
\usepackage{cite}

%%---------------------------------------------------------------------------%%
%% DEFINE SPECIFIC ENVIRONMENTS HERE
%%---------------------------------------------------------------------------%%
%\newcommand{\elfit}{\ensuremath{\operatorname{Im}(-1/\epsilon(\vq,\omega)}}
%\msection{}-->section commands
%\tradem{}  -->add TM subscript to entry
%\ucatm{}   -->add trademark footnote about entry

%%---------------------------------------------------------------------------%%
%% BEGIN DOCUMENT
%%---------------------------------------------------------------------------%%
\begin{document}

%%---------------------------------------------------------------------------%%
%% OPTIONS FOR NOTE
%%---------------------------------------------------------------------------%%

\toms{Distribution}
%\toms{Joe Sixpak/XTM, MS B226}
\refno{CCS-4:02-????(U)}
\subject{Vision and Scope Statements for the Particle Refactor Project}

%-------NO CHANGES
\divisionname{Computer and Computational Sciences}
\groupname{CCS-4:Transport Methods Group}
\fromms{Mike Buksas/CCS-4, MS D409}
\phone{(505)66?--????}
\originator{fml}
\typist{fml}
\date{\today}
%-------NO CHANGES

%-------OPTIONS
%\reference{NPB Star Reimbursable Project}
%\thru{P. D. Soran, XTM, MS B226}
%\enc{list}      
%\attachments{list}
%\cy{list}
%\encas
%\attachmentas
%\attachmentsas 
%-------OPTIONS

%%---------------------------------------------------------------------------%%
%% DISTRIBUTION LIST
%%---------------------------------------------------------------------------%%

\distribution {}

%%---------------------------------------------------------------------------%%
%% BEGIN VISION STATEMENT
%%---------------------------------------------------------------------------%%

\opening

\section*{Vision Statement}

The Particle Refactor Project will produce improved code in the Draco
imc package. Specifically, we seek to reduce the amount of duplicated
code in the Particle class heirachy and simplify the execution of the
transport algorithm. 

This project is being undertaken to simplify the addition of new
features to the transport algorithm. The ease by which specific
features can be added after the refactoring is comlete will be our
criteria for success. This will be done in comparison to the
difficulties encountered in the addition of the surface tracking
capability. 

\newpage
\section*{Scope Statement}

%%
%% Include those of the following elements that apply to your project
%%

\subsection*{Product of the Project}

The product of the Particle Refactor project will be the replacement
of existing code while preserving functionality. {\em No new
  capabilities will be added as part of this project.}

The modifications of the code will likely result in the creation of
new classes and the removal of functionality from the Particle
heriachy of classes. We will use the design in Uncle McFlux, where the
Particle class and the Particle transport algorithm are seperated, as
a starting model for the design.

The design for the refactored code will be based on the anticipation
of several new features: DDIMC, Three dimensional support, Material
Motion and Compton Scattering. Our goal is code which will support the
addition of these features with minimum of effort.

\subsection*{Quality of the Product}

The quality of the new code will be measured informally in terms of
the abstract entities known as {\em code smells}. See~\cite[Chapter
3]{fowler00} for the comprehensive list of smells. An examination of
the code of the Particle classes reveals the following bad smells:
{\em Duplicated Code, Long Method}, and {\em Large Class}.  Based on
the anticipation of new features, we also conclude that the Particle
class suffers from {\em Divergent Change}, that is, it has too many
reasons to change with the addition of new features. Based on the
experience of integrating the Surface Tracking functionality to the
Particle class, we also accuse the broader imc package structure of
suffering from {\em Shotgun Surgery}, that is a new feature prompted
changes in multiple parts of the code.

\subsection*{Delivery Date}

\subsection*{Trade-offs}

E.g., higher product quality implies later delivery date.


\newpage
\section*{Critical Success Factors}

For the Particle Refactor project to be considered a success, we must
be able to recuperate some of the time spent refactoring in time spent
adding new features. This is strictly a software development issue,
and hence is an ordinary problem. The problem acquires a dilemma-like
character becasue it's goal is diffucilt to quantify and can only be
attempted after the project is complete.

To deal with this dilemma, we can consider incorporating the addition
of a selected feature into a broader project, thereby expanding the
scope to include one cycle of ``refactor, then implement''. The
relative ease of the implementation phase can be used to judge the
success of the project.

How will you deal with the critical success factor?
\begin{enumerate}
  \item Resolve it now.
  \item Factor into Plan.
  \item Change Vision/Scope.
  \item Other.
\end{enumerate}

\begin{table}[ht]
  \begin{center}
    \caption{Critical success factors for the Particle Refactor Project.}
    \label{tab:critical-success}
    \begin{tabular}{|c|c|c|c|} 
    \hline
                       & Problem or & Strategy  &          \\
    Factor             &  Dilemma   & (1,2,3,4) & Comments \\ 
    \hline\hline
    New method is bad. &  Dilemma   &    2      & probably doesn't
                                                  resolve bnd layers \\
    \hline
    \end{tabular}
  \end{center}
\end{table}


\newpage
\section*{Risk Assessment}

Identify risks.  For each, estimate the likelihood (1-10) of
occurrence and the impact (1-10) if it happens.  The product of these
two values is a measure of the attention the risk deserves.

\begin{table}[ht]
  \begin{center}
    \caption{Risk Assessment for the Particle Refactor Project.}
    \label{tab:risk}
    \begin{tabular}{|c|c|c|c|c|} 
    \hline
    Risk               & Likelihood & Impact & Importance & Contingency\\ 
    \hline\hline
    Method doesn't     &    8       &   9    &    72      & if it's
                                                            faster,
                                                            combine
                                                            with \\ 
    resolve bnd layers &            &        &            & old
                                                            method; if
                                                            not,
                                                            abandon \\
    \hline
    \end{tabular}
  \end{center}
\end{table}


\newpage
\section*{Project Tracking}

Track actual results and performance with plans.

   Did you change the vision or scope of your project? 
   
   How well did you identify critical success factors?  What lessons
   have you learned?
 
   What risks did your project realize?  How well did you estimate the
   likelihood and impact of the realized risks?  Did you encounter
   unforseen risks?  What lessons have you learned?
 
   How well were you able to estimate delivery dates?  What lessons
   have you learned?

   What corrective action was taken?
   
   How will you improve the vision and scope statements for future
   projects?

\newpage

\bibliography{cs}
\bibliographystyle{plain}

\closing
\end{document}

%%---------------------------------------------------------------------------%%
%% end of particle_transport_refactor.tex
%%---------------------------------------------------------------------------%%

