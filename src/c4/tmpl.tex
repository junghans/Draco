\c LaTeXinfo template file.
\c @> Autodoc template for libc4.tex

\c autodoc outfile=libc4.tex
\c autodoc uplevel=C4
\c autodoc prior_node=Using libc4.a

\node     C4, Namelist, Data Structures, Top
\comment  node-name,  next,  previous,  up
\chapter{Cannonincal Classes for Concurrency Control}
\cindex{libc4.a}
\cindex{Communication Library}

Cannonical Classes for Concurrency Control is a library of C++ classes
designed to ease the process of writing parallel programs on
multicomputers.  A variety of convenience functions and classes are
provided which provide robust interfaces to message passing services.
In some cases the C4 objects and functions provide typesafe interfaces
to simple functionality.  In other cases, they provide abstractions
which the user may subclass and specialize.

C4 is a ``wrapper library''.  See D. Schmidt, C++ Report, for an
excellent introduction to the philosophy of C++ wrapper technology.
Currently C4 provides wrappers for Intel's NX message passing library
for Paragon supercomputers.  However, a port of C4 to MPI is underway
and will be available eventually.

The following sections introduce C4 and document the classes it
provides. 

\begin{menu}
* Using libc4.a::      Linking and using the C4 class library.
<CLASS_LIST_MENU>
\end{menu}

\node Using libc4.a, <FIRST_CLASS>, C4, C4
\section{Using the C4 class library}
\cindex{using C3}

It's easy to use, just jump right in!

<DUMP_CLASSES>
